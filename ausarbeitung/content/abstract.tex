\chapter*{Zusammenfassung}
\begin{center}
  \DocumentThesisTitle
  \\
  \vspace{.3cm}
  \DocumentAuthorPrename\ \DocumentAuthorName ; \DocumentSecondAuthorPrename\ \DocumentSecondAuthorName
\end{center}
\ac{KI} erreicht immer weitere Einsatzgebiete, so auch die Lehre im Hochschulkontext.
Aktuelle Probleme der Hochschullehre, wie die begrenzte Zeit von Lehrkräften und ihre hohe Auslastung, sowie die Auswertung komplexer Texteinreichungen können von künstlicher Intelligenz adressiert und automatisiert werden.
Herausforderungen für den Einsatz von künstlichen Intelligenzen in der Lehre sind unter anderem die Qualität der Daten, eine Sensibilisierung für künstliche Intelligenzen sowie der Mangel an konkreten intelligenten Anwendungen.
Es besteht die Anforderung die Lehre über \ac{KI} auszubauen. Ein Grund hierfür ist bspw., dass \ac{KI} auch interdisziplinär für Problemlösungen eingesetzt werden kann.
Hierfür sind weitere digitale Lerninhalte im Themenfeld \ac{KI} notwendig, die auch unterschiedliche Zielgruppen adressieren. Zugriffsdaten werden in großen Mengen in bestehenden Systemen gesammelt. Mit Lernmanagementsystemen, die an Hochschulen bereits im Einsatz sind, werden Zugriffsdaten auch von Lernenden gesammelt.
Zugriffsdaten können als Trainingsdaten für die Entwicklung von intelligenten Systemen verwendet werden.
Die Frage lautet, ob Zugriffsdaten von Studierenden für die Entwicklung einer \ac{KI} im Hochschultext eingesetzt werden können.
Ein Ansatz verwendet einige Zugriffsdaten des Lernmanagementsystems Moodle, um vorherzusagen, ob Studierende einen Kurs bestehen oder nicht.
Die entwickelte \ac{KI} kann 97 \% der Studierenden, die den Kurs bestehen, und 78 \% der Studierenden, die den Kurs nicht bestehen, vorhersagen.
Insgesamt sind wir positiv gegenüber dem Einsatz von künstlichen Intelligenzen in der Hochschullehre gestimmt.
Wir finden, dass die Datentransparenz eine große Rolle spielen wird und die Verwendung von Daten offen und transparent kommuniziert werden muss.

\newpage

\chapter*{Abstract}
\begin{center}
  \DocumentThesisTitle
  \\
  \vspace{.3cm}
  \DocumentAuthorPrename\ \DocumentAuthorName ; \DocumentSecondAuthorPrename\ \DocumentSecondAuthorName
\end{center}
Artificial intelligence (AI) is reaching more and more fields - including teaching in a university context.
Current problems of university teaching, such as the limited time of teachers and their high workload, as well as the evaluation of complex text submissions can be addressed and automated by artificial intelligence.
Challenges to the use of artificial intelligences in teaching include the quality of data, an awareness of artificial intelligences, and the lack of concrete intelligent applications.
There is a requirement to expand teaching about AI. One reason for this is, for example, that AI can also be used interdisciplinary for solving problems.
This requires further digital learning content in the field of AI, which also address different audiences. 
Access data is collected in large quantities in existing systems. With Learning Management Systems, which are already used at universities, access data is also collected from students.
Access data can be used as training data for the development of intelligent systems. The question is whether student access data can be used to develop an AI in higher education text. 
One approach uses some access data from the Moodle Learning Management System to predict whether students pass or fail a course. The developed AI can predict 97 \% of students who pass the course and 78 \% of students who fail the course.
Overall, we are positively inclined toward the use of artificial intelligence in higher education.
We find, that data transparency will play a major role and the use of data has to be communicated openly and transparently.

