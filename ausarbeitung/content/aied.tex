Artificial Intellgene in Education (AIED)

Paper beschäftigt sich damit den State of the Art anzuschauen (vergangenen 25 Jahre )um rückschlüsse auf wie sich das
thema in den Nächsten Jahren entwickelnt wird (auch 25). Die Aktuellen trends werden dabei Interpretier in einer Utopie und Dystopie
Zum Schluss sollen noch sieben wichtige Risikien Herausforderungen und Chanchen ermittelt werden für AIED um erfolgreich zu sein.

Die Autoren wissen jedoch auch das es nicht möglich sein wird etwas vorherzusagen, jedoch soll das Paper in zukunft dazu dienen um zu wissen was man sich früher bei dem Thema gedacht hat.


\subsubsection*{Current Trends and Developments}

Nehmen AIED gibt es viele Faktoren ausßerhalb von AIED die AIED beeinflussen.
GErman Computer Sciende Association hat fünf große Ziele definiert für die Computer wissenschaft. In dieser
liste stehen auch herausforderungen die relevant für AIED sind. Zum Beispiel der Schutz von Lerndaten der Nutezrverhalten.

Viel Einfluss wird auch die Big Data und Industrie 4.0 haben. Da es unumstreitbar ist, das AIED viele Daten brauch um besser zu funktionieren

\subsubsection{Educational trends and Developments Relevant to AIED}
Die Dedactic ist ein nicht schnell änders Feld.

TODO:

Unterschiedliche Kulturen lernen Anders, daher ist es schwierig eine Software zu haben die in vielen Ländern eingesetzt werden kann.
Außerdem ist es dadruch schwiierig eine allgegenwörtige, universelle Interaktionsmethode mit der Software/Hardware herzustellen,
welche nötig ist damit Lehrende diese in ihere Lehre einbinden können. 
Ein Weitres Problem ist es, dass Lerherende, die von Ihnen verwendete Technologie verstehen und deren Vorteile kennen müssen,
dies passiert jedoch in den seltesten Fällen. Des Weitren ist die Schulinfrastruktur nicht ausreichend genug vorhanden um verschiedene technologien einsetzen pädagogisch sinnvoll einsetzen zu können.


\subsubsection{Two Szenarios}
Das Paper nennt zwei extreme Szenarien die in 25 Jahren eintretten könnten und die uns vorauge führen sollen, in welche Rcithungen AIED gehen kann
\subsubsection{Utopie}
Zwei personen unterhalten sich darüber wie hilfreich deren verwendete Technologie ihnen bei ihrem
lernen geholfen hat, wie sie dadurch inhalte besser verstanden haben, Sich die anwendungen an deren Lernart angepasst hat und das sie sie motivieren mit spielen und sie Spaß beim lernen haben, sowie auch am Wochendende Hilfe von Digitalen Tutoren haben können und Aufgaben kontrollieren lassen
und das die Lehrere keinen einsicht in die Daten haben und deren Daten nur verwendet wird um ihnen zu helfen und um anonymisiert weitergeleitet werden um die tools zu verbessern. Und das deren Daten benutzt werden um die Tools zu verbessern.
\subsubsection{Distopie}
Zwei Personen sitzen ind er Bücherrei und bevorzugen bücher anstelle von Tools. Das Problem ist hier, das die Tools nicht für deren Land ausgerichtet sind und deren lehrer diese nicht benutzen. 
Des Weiteren vermuten die Personen, das die Lehere auch besorgt um deren Job ist. Denn in den letzten 10 Jahren haben 50 \% der lehrer von Mateh und Wissenschaftsfächern ihren Job verloren.
Die Tools die benutzt werden besondern in Ethk fächern oder ähnliches sollte mit vorsicht genutzt werden, da diese die Antworten der Nutzer speichert und weiterverkauft. Es gibt Personen die daher keinen Job gefunden haben.
Daher werden auch viele Tools und Foren von wenig Personen benutzt und man bekommt kaum Hilfe.

\subsubsection{Discussion & Conclusion}
TODO: