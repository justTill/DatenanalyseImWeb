Artificial Intelligence in Education (AIED)

Paper beschäftigt sich damit den State of the Art anzuschauen (vergangenen 25 Jahre )um rückschlüsse auf wie sich das
thema in den Nächsten Jahren entwickelnt wird (auch 25). Die Aktuellen trends werden dabei Interpretier in einer Utopie und Dystopie
Zum Schluss sollen noch sieben wichtige Risikien Herausforderungen und Chanchen ermittelt werden für AIED um erfolgreich zu sein.

Die Autoren wissen jedoch auch das es nicht möglich sein wird etwas vorherzusagen, jedoch soll das Paper in zukunft dazu dienen um zu wissen was man sich früher bei dem Thema gedacht hat.


\subsubsection*{Current Trends and Developments}

Nehmen AIED gibt es viele Faktoren ausßerhalb von AIED die AIED beeinflussen.
GErman Computer Sciende Association hat fünf große Ziele definiert für die Computer wissenschaft. In dieser
liste stehen auch herausforderungen die relevant für AIED sind. Zum Beispiel der Schutz von Lerndaten der Nutezrverhalten.

Viel Einfluss wird auch die Big Data und Industrie 4.0 haben. Da es unumstreitbar ist, das AIED viele Daten brauch um besser zu funktionieren

\subsubsection{Educational trends and Developments Relevant to AIED}
Die Dedactic ist ein nicht schnell änders Feld.

Unterschiedliche Kulturen lernen Anders, daher ist es schwierig eine Software zu haben die in vielen Ländern eingesetzt werden kann.
Außerdem ist es dadruch schwiierig eine allgegenwörtige, universelle Interaktionsmethode mit der Software/Hardware herzustellen,
welche nötig ist damit Lehrende diese in ihere Lehre einbinden können. 
Ein Weitres Problem ist es, dass Lerherende, die von Ihnen verwendete Technologie verstehen und deren Vorteile kennen müssen,
dies passiert jedoch in den seltesten Fällen. Des Weitren ist die Schulinfrastruktur nicht ausreichend genug vorhanden um verschiedene technologien einsetzen pädagogisch sinnvoll einsetzen zu können.


\subsubsection{Educational technology trends and Developments}
Es wurde herausgefunden, das es trends gibt die ein Impact haben auf AIED: Bring Your Own device, Wearable Technology, Adaptiv Learning Technologoes and IOT

Personalisieren des lernen durch Learning Analytics (Data driven analytics)

Systeme Kosten und man brauch wahrscheinlich verschiedene Systeme für verschiedenen Kunden. Soll AIED Open Source sein?




\subsubsection{Two Szenarios}
Das Paper nennt zwei extreme Szenarien die in 25 Jahren eintretten könnten und die uns vorauge führen sollen, in welche Rcithungen AIED gehen kann
\subsubsection{Utopie}
Zwei personen unterhalten sich darüber wie hilfreich deren verwendete Technologie ihnen bei ihrem
lernen geholfen hat, wie sie dadurch inhalte besser verstanden haben, Sich die anwendungen an deren Lernart angepasst hat und das sie sie motivieren mit spielen und sie Spaß beim lernen haben, sowie auch am Wochendende Hilfe von Digitalen Tutoren haben können und Aufgaben kontrollieren lassen
und das die Lehrere keinen einsicht in die Daten haben und deren Daten nur verwendet wird um ihnen zu helfen und um anonymisiert weitergeleitet werden um die tools zu verbessern. Und das deren Daten benutzt werden um die Tools zu verbessern.
\subsubsection{Distopie}
Zwei Personen sitzen ind er Bücherrei und bevorzugen bücher anstelle von Tools. Das Problem ist hier, das die Tools nicht für deren Land ausgerichtet sind und deren lehrer diese nicht benutzen. 
Des Weiteren vermuten die Personen, das die Lehere auch besorgt um deren Job ist. Denn in den letzten 10 Jahren haben 50 \% der lehrer von Mateh und Wissenschaftsfächern ihren Job verloren.
Die Tools die benutzt werden besondern in Ethk fächern oder ähnliches sollte mit vorsicht genutzt werden, da diese die Antworten der Nutzer speichert und weiterverkauft. Es gibt Personen die daher keinen Job gefunden haben.
Daher werden auch viele Tools und Foren von wenig Personen benutzt und man bekommt kaum Hilfe.

\subsubsection{Discussion und Conclusion}
Beide Szenarien folgen den aktuellen Technologie Trends und Herausforderungen, mit denen sich AIED Technologien auseinandersetzen müssen

Challenges: Intercultural and Global Dimensions. Auch wenn in Zukunft es theoretisch möglich ist jedem Schüler/Student AIED Systeme zukommen zu lassen.
 Bleibt die Herausforderungen das es unterschiedliche Kulturen gibt und jede dieser Kulturen auf eine andere Art und weise das Wissen vermittelt. Dies Liegt zu einem an der Sprache, 
 verschiedene Lehrpläne, die pädagogische Kultur. Dies trifft vorallem auf, wenn typische Schüler-Lehere Interaktionen auftreten, so wie es bei vielen AIED der fall ist /sein soll.

 Daher ist es notwendig die Kulturellen verschiedeneheiten genauer zu erforschen und AIED Systeme genau zu designen und zu Implementieren, damit mit den Gesammelten Daten nichts falsches angestellt werden kann.

 Challenges: Practial Impact. Im Moment gibt es schon Kooperationen und benutzte AIED Systeme, dies muss in Zukunft verstärkt werden. Besonders in Schulen. Dafür wird jedoch personal in den Schulen benötigt
 falls es Probleme mit dem System gibt, oder Personen Hilfe beim Umgang mit diesem brauche, sowie eine dauerhafte erreichbarkeit und benutzbarkeit.


 Challenges: Privacy. Wenn privat Firmen AIED Systeme bereitstellen, spielt daten privatspähre eine besonder Rolle. Es gibt bereits Gesetzte und Regelungen in und zwischen Ländern zu regelung der Datenprivatsphäre, diese 
 werden in Zukunft noch eine Wichtige Rolle spielen. Durch die Strikten regelen wird es auch schwer gewinn mit commerciellen AIED Systemen zu machen, denn wenn man kein Gewinn mit den Daten machen kann. Webung kann man schlecht schalten, das stört den Lernfluss,
 Zu Hohe licens kosten, können sich einige Schulen und Lehrinstitutionen nicht leisten. Wenn mit den LernDaten gewinne erszielt werden sollen, müssen Transparente und Kkalr definierte und kontrollierte Regeln in Kraft treten.


 Challenges: Interaction Methods. Es muss genau erforscht werden wie Menschen mit den Computersytemen umgehen und versuchen festzustellen wie dies in zukunft aussehen wird, da sich diese stetig ändert und es nicht bringt
 AIED Systeme für heute zu entwicklen, sondern für morgen.


 Challenges: Collaboration at scale: Derzeit gibt es schon Online Curese die Weltwerit genutzt werden, AIED Systeme sollten daher auch die möglichkeit haben Weltweit eingesetzt zu werden.

 Challenges: Effectiveness in multiple Domains: ...?? TODO:


 Challenges: Role of AIED in educational technology: hmmmm ??
