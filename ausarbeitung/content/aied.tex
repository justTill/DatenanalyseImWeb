
Professor Niels Pinkwart ist bei der Humbolt-Universität in Berlin als Leiter für \glqq{} Didaktik der Informatik / Informatik und Gesellschaft\grqq{} tätig. 
Er hat über 250 Publikationen veröffentlicht. Darunter auch den Artikel \glqq{} Another 25 Years of AIED? Challenges and Opportunities for Intelligent Educational Technologies of the Future\grqq{}, der im Jahre 2016 im \glqq International Journal of Artificial Intelligence in Education\grqq{} veröffentlicht wurde.
Dieser Artikel wurde laut der Webseite Springer 45 mal von anderen Autoren zitiert. \footnote{\url{https://link.springer.com/article/10.1007/s40593-016-0099-7}}
\\ \noindent
In dem Artikel wird versucht die aktuellen Trends und Entwicklungen in der Informatik, der Lehre und der Lehre mit Technologien zu analysieren. Dies hilft dabei rückschlüsse auf die Weiterentwicklung
von \ac{AIED} (dt. Künstliche Intellgenz in der Lehre) in den nächsten 25 Jahren schließen zu können. Basierend auf disen Trends werden zwei mögliche Zukunftszenarien vorgestell.
Es gibt eine Utopie und eine Dystopie. Zu letzte werden mögliche Herausforderungen und Chancen für die Entwicklung von \ac{AIED} Systeme identifiziert, die in den nächsten 25 Jahren entstehen können. 
Da es nicht möglich ist 25 Jahre in die Zukunft zu schauen, kann der Artikel in Zukunft benutzt werden um zu schauen wie sich \ac{AIED} entwickelt hat im vergleich zu dem was gedacht wurde.
\\ \noindent
Professor Niels nennt sieben Herausforderungen, die sich aus der Untersuchung und den Szenarien hervorgehen, sind im folgenden gennat und erläutert.
\begin{enumerate}
    \item intercultural and global dimensions (dt. Interkulturelle und Gloable Dimensionen) \\ 
    Lernsoftware kann in einem kulturellen Kontext sehr gut funktionieren in ein einem anderen nicht. Dies liegt an vielen verschiedenen Faktoren, der Sprache, den verschiedenen Inhalten der Lehrpläne,
    der Lehrkultur und die unterschiedlichen Art und weisen, wie die Interaktion zwischen Lernenden und Lehrender ist. Es kann vermuted werden, dass in 25 Jahren überalle auf der Erde Internet verfügbar sein,
    und damit möglich jedem Lernenden AIED Systeme zu verfügung gestellt werden kann. Es muss daher in dem Bereich Interkulturelle lehre geforscht werden, damit AIED Systeme entwicklen werden können, die allen Lernenden hilft. 
    
    \item practical impact (dt. praktische Auswirkungen) \\
    Es gibt bereits einige AIED Systeme in Schulen eingesetzt werden, diese passiert oft mit kommerziellen Initiativen innerhalb des AIED-Bereichs.
    Diese Verbindungen zu Schulen sind zwar wichtig, um den Einfluss von AIED Systemen auf die Bildung weiter zu erhöhen. Dennoch werden in Forschungssystemen gewisse Interessensgruppen wie Lehrkäfte und Schulen oft nicht eingebunden.
    Besonders in Schulen ist die Frage, nach technischem Support, Fehlerbehebungen, Stabilität und kontinuierliche Verfügbarkeit sehr wichtig und wird häufiger nicht ausreichend beachtet. 
    Daher besteht die Herausforderung darin, wege zu finden die über Forschungsysteme hinausgehen und dauerhaft bestehen.

    \item privacy (dt. Privatspähre) \\
    Dadruch das AIED Systeme als Begeleiter fürs Lebenslange lernen dienen sollen, werden viele Daten benötigt und gesammelt.
    Diese müssen geschützt werden, damit diese z. B. nicht genutzt werden um möglicherweise Kanidaten bei einer Bewerbung im vorhinein auzusortieren, sollten die Daten von Firmen verkauft werden.
    Diese kann passieren, wenn AIED Systeme von privaten Unternehmen entwickelt werden. Daher spielt der Datenschutz eine sehr wichtige Rolle. Die Regeln könnten sich in Zukunft verschärfen.
    Dies ist zwar für den Benutzt gut, stellt jedoch für die Unternehmen eine weitere Herausforderung da. Diese möchten Gewinne erzielen, diese werden häuzutage oft mit Daten der Nutzer erzielt.
    Ist dies durch die strikteren Regeln nicht mehr möglich, muss eine andere möglichkeit gesucht werden. Eine der Möglichkeiten währe, Werbung schalten, dies würde aber den Lernfluss stören.
    Eine andere Möglichkeit wäre die Lizenzierung, dann könnte jedoch das Problem enstehen, dass sich einige Institutionen sich diese Software nicht leisten können.
    Sollte also mit den Daten der Lernen Gewinn erzielt werden, werden klare und für den Nutzer Transparente Regeln nötig.

    \item Interaction methods (dt. Interaktionsmethoden) \\
    Die vierte Herausforderung sind die Interaktionsmethoden. Durch die entwicklung von Mensch-Computer-Interaktion ist die Interaktion der Lernenden und den AIED Systemen einfacher und intuitiver als früher.
    Daher ist es möglich, dass sich dieser Trend weiter fortsetzt. Es muss überlegt werden, wie die Lernenden in Zukunft mit AIED Systemen umgehen sollen.
    Es muss für die Lernednen möglich sein, leicht informationen einzugeben und bequemes Feedback zurückzubekommen. Die schwierigkeit besteht darin, die Eingaben, die über unterschiedliche Kanäle (z. B. Sprache oder Text) besonders gut und ohne Störungen zu empfangen, damit diese Analysiert werden können.


    \item collaboration at scale (dt. Zusammenarbeit im großen Maßstab) \\
    Es gibt viele Gloable online Kurse, die tausenden von Lernenden erreichen. Diese setzen jedoch nur auf einfache Bildungstechnologien wie Video, Texte, Multiple-Choice-Tests und Diskussionsforen.
    Möglichkeiten und einige Grundlagen für das gloabe gemeinsame Lernen mit intelligenten Technicken, wie eine personalisierte Anleitung und zum zusammenarbeiten sind schon vorhanden. 
    Damit Lehrkräften und Teilnehmern an großen Online-Kursen besser unterstützt und gefördert werden, muss an diesen techniken noch weiter geforscht werden,
    
    \item effectiveness in multiple domains (dt. Effektivität in verschiedenen Bereichen) \\
    
    \item role of ac{AIED} in educational technology (dt. Rolle von \ac{AIED} in Bildungstechnologien) \\
    In vielen Bereichen wie z. B. die Mensch-Computer-Interaktion gibt es bereits globale Konferenzen (CHI-Konferenz) und Veranstaltungen die einen gewissen Prestige vorweisen und die weitere entwicklung des Bereichs stark verändern können.
    Im Bereich der Bildungtechnologien gibt es derzeitig nichts vergleichbares. Es gibt jedoch eine große Bandbreite an Veranstaltungen und damit viele Richtungen in welche die Forschung gehen kann. Dies ist erstmal nichts negatives, da sich so ein Bereich auch weiterentwickelt.
    Die Herausforderung besteht nun darin eine globale Gemeinschaft der Forschung im Bereich der Bildungstechnologie aufzubauen umd so starte Fragmentierung zu vermeiden.
    Eine zu Starke Fragmentierung kann dazu führen, das sich ein Bereich nicht stark und gut weiterentwickelt und damit keine Gesamtwirkung nachweisen kann.
\end{enumerate} 


\begin{comment}
Die Gesellschaft für Informatiker hat Herausforderungen für den Bereich Informatik definiert. Diese gelten selbstverständlich auch für \ac{AIED} Systeme.
Bei den Herausforderungen handelt es sich um die Bewahrung und Archivierung des digitalen Kulturerbes, ermöglichung eines sicheren, privaten, freien und vertrauensvollen Nutzung des Internets,
Bewältigung von Risikin in Globalen IT Infrastrukturen, Entwicklung von Interaktionsmöglichkeiten, die es allen Menschen ermöglicht von IT Systemen zu profitieren, sowie die Gewährleitungs das IT Systeme dauerhaft verfügbar sind.
weitere 
\end{comment}
    
\subsection{Notizen}

\subsubsection*{Current Trends and Developments}

\subsubsection{Educational trends and Developments Relevant to AIED}
Die Dedactic ist ein nicht schnell änders Feld.

Unterschiedliche Kulturen lernen Anders, daher ist es schwierig eine Software zu haben die in vielen Ländern eingesetzt werden kann.
Außerdem ist es dadruch schwiierig eine allgegenwörtige, universelle Interaktionsmethode mit der Software/Hardware herzustellen,
welche nötig ist damit Lehrende diese in ihere Lehre einbinden können. 
Ein Weitres Problem ist es, dass Lerherende, die von Ihnen verwendete Technologie verstehen und deren Vorteile kennen müssen,
dies passiert jedoch in den seltesten Fällen. Des Weitren ist die Schulinfrastruktur nicht ausreichend genug vorhanden um verschiedene technologien einsetzen pädagogisch sinnvoll einsetzen zu können.


\subsubsection{Educational technology trends and Developments}
Es wurde herausgefunden, das es trends gibt die ein Impact haben auf AIED: Bring Your Own device, Wearable Technology, Adaptiv Learning Technologie and IOT

Personalisieren des lernen durch Learning Analytics (Data driven analytics)

Systeme Kosten und man brauch wahrscheinlich verschiedene Systeme für verschiedenen Kunden. Soll AIED Open Source sein?




\subsubsection{Two Szenarios}
Das Paper nennt zwei extreme Szenarien die in 25 Jahren eintretten könnten und die uns vorauge führen sollen, in welche Rcithungen AIED gehen kann

\subsubsection{Utopie}
Zwei personen unterhalten sich darüber wie hilfreich deren verwendete Technologie ihnen bei ihrem
lernen geholfen hat, wie sie dadurch inhalte besser verstanden haben, Sich die anwendungen an deren Lernart angepasst hat und das sie sie motivieren mit spielen und sie Spaß beim lernen haben, sowie auch am Wochendende Hilfe von Digitalen Tutoren haben können und Aufgaben kontrollieren lassen
und das die Lehrere keinen einsicht in die Daten haben und deren Daten nur verwendet wird um ihnen zu helfen und um anonymisiert weitergeleitet werden um die tools zu verbessern. Und das deren Daten benutzt werden um die Tools zu verbessern.


\subsubsection{Distopie}
Zwei Personen sitzen ind er Bücherrei und bevorzugen bücher anstelle von Tools. Das Problem ist hier, das die Tools nicht für deren Land ausgerichtet sind und deren lehrer diese nicht benutzen. 
Des Weiteren vermuten die Personen, das die Lehere auch besorgt um deren Job ist. Denn in den letzten 10 Jahren haben 50 \% der lehrer von Mateh und Wissenschaftsfächern ihren Job verloren.
Die Tools die benutzt werden besondern in Ethk fächern oder ähnliches sollte mit vorsicht genutzt werden, da diese die Antworten der Nutzer speichert und weiterverkauft. Es gibt Personen die daher keinen Job gefunden haben.
Daher werden auch viele Tools und Foren von wenig Personen benutzt und man bekommt kaum Hilfe.

\subsubsection{Discussion und Conclusion}
Beide Szenarien folgen den aktuellen Technologie Trends und Herausforderungen, mit denen sich AIED Technologien auseinandersetzen müssen

Challenges: Intercultural and Global Dimensions. Auch wenn in Zukunft es theoretisch möglich ist jedem Schüler/Student AIED Systeme zukommen zu lassen.
 Bleibt die Herausforderungen das es unterschiedliche Kulturen gibt und jede dieser Kulturen auf eine andere Art und weise das Wissen vermittelt. Dies Liegt zu einem an der Sprache, 
 verschiedene Lehrpläne, die pädagogische Kultur. Dies trifft vorallem auf, wenn typische Schüler-Lehere Interaktionen auftreten, so wie es bei vielen AIED der fall ist /sein soll.

 Daher ist es notwendig die Kulturellen verschiedeneheiten genauer zu erforschen und AIED Systeme genau zu designen und zu Implementieren, damit mit den Gesammelten Daten nichts falsches angestellt werden kann.

 Challenges: Practial Impact. Im Moment gibt es schon Kooperationen und benutzte AIED Systeme, dies muss in Zukunft verstärkt werden. Besonders in Schulen. Dafür wird jedoch personal in den Schulen benötigt
 falls es Probleme mit dem System gibt, oder Personen Hilfe beim Umgang mit diesem brauche, sowie eine dauerhafte erreichbarkeit und benutzbarkeit.


 Challenges: Privacy. Wenn privat Firmen AIED Systeme bereitstellen, spielt daten privatspähre eine besonder Rolle. Es gibt bereits Gesetzte und Regelungen in und zwischen Ländern zu regelung der Datenprivatsphäre, diese 
 werden in Zukunft noch eine Wichtige Rolle spielen. Durch die Strikten regelen wird es auch schwer gewinn mit commerciellen AIED Systemen zu machen, denn wenn man kein Gewinn mit den Daten machen kann. Webung kann man schlecht schalten, das stört den Lernfluss,
 Zu Hohe licens kosten, können sich einige Schulen und Lehrinstitutionen nicht leisten. Wenn mit den LernDaten gewinne erszielt werden sollen, müssen Transparente und klar definierte und kontrollierte Regeln in Kraft treten.


 Challenges: Interaction Methods. Es muss genau erforscht werden wie Menschen mit den Computersytemen umgehen und versuchen festzustellen wie dies in zukunft aussehen wird, da sich diese stetig ändert und es nicht bringt
 AIED Systeme für heute zu entwicklen, sondern für morgen.


 Challenges: Collaboration at scale: Derzeit gibt es schon Online Curese die Weltwerit genutzt werden, AIED Systeme sollten daher auch die möglichkeit haben Weltweit eingesetzt zu werden.

 Challenges: Effectiveness in multiple Domains: ...?? TODO:


 Challenges: Role of AIED in educational technology: hmmmm ??
