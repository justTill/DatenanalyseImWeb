
Professor Niels Pinkwart ist bei der Humbolt-Universität in Berlin als Leiter für \glqq{} Didaktik der Informatik / Informatik und Gesellschaft\grqq{} tätig. 
Er hat über 250 Publikationen veröffentlicht. Darunter auch den Artikel \glqq{} Another 25 Years of AIED? Challenges and Opportunities for Intelligent Educational Technologies of the Future\grqq{}, der im Jahre 2016 im \glqq International Journal of Artificial Intelligence in Education\grqq{} veröffentlicht wurde.
Dieser Artikel wurde laut der Webseite Springer 45 mal von anderen Autoren zitiert. \footnote{\url{https://link.springer.com/article/10.1007/s40593-016-0099-7}, zuletzt aufgerufen am 10.07.2022}
\\ \noindent
In dem Artikel wird versucht die aktuellen Trends und Entwicklungen in der Informatik, der Lehre und der Lehre mit Technologien zu analysieren. Dies hilft dabei rückschlüsse auf die Weiterentwicklung
von \ac{AIED} (dt. Künstliche Intellgenz in der Lehre) in den nächsten 25 Jahren schließen zu können. Basierend auf disen Trends werden zwei mögliche Zukunftszenarien vorgestell.
Es gibt eine Utopie und eine Dystopie. Zu letzte werden mögliche Herausforderungen und Chancen für die Entwicklung von \ac{AIED} Systeme identifiziert, die in den nächsten 25 Jahren entstehen können. 
Da es nicht möglich ist 25 Jahre in die Zukunft zu schauen, kann der Artikel in Zukunft benutzt werden um zu schauen wie sich \ac{AIED} entwickelt hat im vergleich zu dem was gedacht wurde. \cite[S. 1f]{Pinkwart.2016}
\\ \noindent
Professor Niels nennt sieben Herausforderungen, die sich aus der Untersuchung und den Szenarien hervorgehen, sind im folgenden gennat und erläutert.
\begin{enumerate}
    \item intercultural and global dimensions (dt. Interkulturelle und Gloable Dimensionen) \\ 
    Lernsoftware kann in einem kulturellen Kontext sehr gut funktionieren in ein einem anderen nicht. Dies liegt an vielen verschiedenen Faktoren, der Sprache, den verschiedenen Inhalten der Lehrpläne,
    der Lehrkultur und die unterschiedlichen Art und weisen, wie die Interaktion zwischen Lernenden und Lehrender ist. Es kann vermuted werden, dass in 25 Jahren überalle auf der Erde Internet verfügbar sein,
    und damit möglich jedem Lernenden AIED Systeme zu verfügung gestellt werden kann. Es muss daher in dem Bereich Interkulturelle lehre geforscht werden, damit AIED Systeme entwicklen werden können, die allen Lernenden hilft. \cite[S. 9f]{Pinkwart.2016} 
    
    \item practical impact (dt. praktische Auswirkungen) \\
    Es gibt bereits einige AIED Systeme in Schulen eingesetzt werden, diese passiert oft mit kommerziellen Initiativen innerhalb des AIED-Bereichs.
    Diese Verbindungen zu Schulen sind zwar wichtig, um den Einfluss von AIED Systemen auf die Bildung weiter zu erhöhen. Dennoch werden in Forschungssystemen gewisse Interessensgruppen wie Lehrkäfte und Schulen oft nicht eingebunden. 
    Besonders in Schulen ist die Frage, nach technischem Support, Fehlerbehebungen, Stabilität und kontinuierliche Verfügbarkeit sehr wichtig und wird häufiger nicht ausreichend beachtet. 
    Daher besteht die Herausforderung darin, wege zu finden die über Forschungsysteme hinausgehen und dauerhaft bestehen. \cite[S. 10]{Pinkwart.2016}

    \item privacy (dt. Privatspähre) \\
    Dadruch das AIED Systeme als Begeleiter fürs Lebenslange lernen dienen sollen, werden viele Daten benötigt und gesammelt \cite[S. 10]{Pinkwart.2016}. 
    Diese müssen geschützt werden, damit diese z. B. nicht genutzt werden um möglicherweise Kanidaten bei einer Bewerbung im vorhinein auzusortieren, sollten die Daten von Firmen verkauft werden \cite[S. 9f]{Pinkwart.2016}.
    Diese kann passieren, wenn AIED Systeme von privaten Unternehmen entwickelt werden. Daher spielt der Datenschutz eine sehr wichtige Rolle. Die Regeln könnten sich in Zukunft verschärfen.
    Dies ist zwar für den Benutzt gut, stellt jedoch für die Unternehmen eine weitere Herausforderung da. Diese möchten Gewinne erzielen, diese werden häuzutage oft mit Daten der Nutzer erzielt.
    Ist dies durch die strikteren Regeln nicht mehr möglich, muss eine andere möglichkeit gesucht werden. Eine der Möglichkeiten währe, Werbung schalten, dies würde aber den Lernfluss stören.
    Eine andere Möglichkeit wäre die Lizenzierung, dann könnte jedoch das Problem enstehen, dass sich einige Institutionen sich diese Software nicht leisten können.
    Sollte also mit den Daten der Lernen Gewinn erzielt werden, werden klare und für den Nutzer Transparente Regeln nötig. \cite[S. 10f]{Pinkwart.2016}

    \item Interaction methods (dt. Interaktionsmethoden) \\
    Die vierte Herausforderung sind die Interaktionsmethoden. Durch die entwicklung von Mensch-Computer-Interaktion ist die Interaktion der Lernenden und den AIED Systemen einfacher und intuitiver als früher.
    Daher ist es möglich, dass sich dieser Trend weiter fortsetzt. Es muss überlegt werden, wie die Lernenden in Zukunft mit AIED Systemen umgehen sollen.
    Es muss für die Lernednen möglich sein, leicht informationen einzugeben und bequemes Feedback zurückzubekommen. Die schwierigkeit besteht darin, die Eingaben, die über unterschiedliche Kanäle (z. B. Sprache oder Text) besonders gut und ohne Störungen zu empfangen, damit diese Analysiert werden können. \cite[S. 11]{Pinkwart.2016}


    \item collaboration at scale (dt. Zusammenarbeit im großen Maßstab) \\
    Es gibt viele Gloable online Kurse, die tausenden von Lernenden erreichen. Diese setzen jedoch nur auf einfache Bildungstechnologien wie Video, Texte, Multiple-Choice-Tests und Diskussionsforen.
    Möglichkeiten und einige Grundlagen für das gloabe gemeinsame Lernen mit intelligenten Technicken, wie eine personalisierte Anleitung und zum zusammenarbeiten sind schon vorhanden. 
    Damit Lehrkräften und Teilnehmern an großen Online-Kursen besser unterstützt und gefördert werden, muss an diesen techniken noch weiter geforscht werden. \cite[S. 11]{Pinkwart.2016}
    
    \item effectiveness in multiple domains (dt. Effektivität in verschiedenen Bereichen) \\
    Die sechte Herausforderung besteht darin, dass AIED Systeme auch in anderen Bereich aktiv eingesetzt werden, die heutzutage nicht vertreten sind. Solche Bereiche sind die, die nicht genau definiert sind.
    Heutzutage gibt es viele AIED Systemen in Bereichen, wo die Domäne bekannt und gut modeliert werden kann z. B. Geometrie.
    Damit in Zukunft auch Ethikaufsätze oder ein Virtual-Reality-Chirurgietraining mithilfe von AIED Systemen abgedeckt werden können, wird weiter forschung und Entwicklung in den wissenschaftlichen Bereichen wie der Technologie zur Verarbeitung natürlicher Sprache oder der Wissensdarstellung benötigt. \cite[S. 11f]{Pinkwart.2016}
     
    
    \item role of ac{AIED} in educational technology (dt. Rolle von \ac{AIED} in Bildungstechnologien) \\
    In vielen Bereichen wie z. B. die Mensch-Computer-Interaktion gibt es bereits globale Konferenzen (CHI-Konferenz) und Veranstaltungen die einen gewissen Prestige vorweisen und die weitere entwicklung des Bereichs stark verändern können.
    Im Bereich der Bildungtechnologien gibt es derzeitig nichts vergleichbares. Es gibt jedoch eine große Bandbreite an Veranstaltungen und damit viele Richtungen in welche die Forschung gehen kann. Dies ist erstmal nichts negatives, da sich so ein Bereich auch weiterentwickelt.
    Die Herausforderung besteht nun darin eine globale Gemeinschaft der Forschung im Bereich der Bildungstechnologie aufzubauen umd so starte Fragmentierung zu vermeiden.
    Eine zu Starke Fragmentierung kann dazu führen, das sich ein Bereich nicht stark und gut weiterentwickelt und damit keine Gesamtwirkung nachweisen kann. \cite[S. 12]{Pinkwart.2016}
\end{enumerate} 