
Professor Niels Pinkwart ist bei der Humbolt-Universität in Berlin als Leiter für \glqq{}Didaktik der Informatik / Informatik und Gesellschaft\grqq{} tätig.
Er hat über 250 Publikationen veröffentlicht. Darunter auch das Paper \glqq{}Another 25 Years of AIED? Challenges and Opportunities for Intelligent Educational Technologies of the Future\grqq{}, das im Jahre 2016 im \glqq International Journal of Artificial Intelligence in Education\grqq{} veröffentlicht wurde.
Dieser Paper wurde laut der Webseite Springer 45 mal von anderen Autoren zitiert. \footnote{\url{https://link.springer.com/article/10.1007/s40593-016-0099-7}, zuletzt aufgerufen am 10.07.2022}
\\ \noindent
In dem Paper wird versucht die aktuellen Trends und Entwicklungen in der Informatik, der Lehre und der Lehre mit Technologien zu analysieren. Dies hilft dabei, Rückschlüsse auf die Weiterentwicklung
von \ac{AIED} (dt. Künstliche Intellgenz in der Lehre) in den nächsten 25 Jahren schließen zu können. Basierend auf diesen Trends, werden zwei mögliche Zukunftszenarien vorgestellt.
Es gibt eine Utopie und eine Dystopie. Abschließend werden mögliche Herausforderungen und Chancen für die Entwicklung von \ac{AIED} Systeme identifiziert, die in den nächsten 25 Jahren entstehen können.
Da es nicht möglich ist 25 Jahre in die Zukunft zu schauen, kann der Artikel in Zukunft benutzt werden um zu übeprüfen, wie sich \ac{AIED} im Gegensatz zur Erwartung entwickelt hat. \cite[S. 1f]{Pinkwart.2016}
\\ \noindent
\\ \noindent
Aus der Untersuchung der aktuellen Trends und der Szenarien lassen sich sieben Herausforderung ableiten, diese sind im folgenden aufgelistet und erläutert.
\begin{enumerate}
      \item \textbf{Intercultural and global dimensions (dt. Interkulturelle und Gloable Dimensionen)} \\
            Lernsoftware kann in einem kulturellen Kontext sehr gut funktionieren, in einem anderen möglicherweise nicht. Dies liegt an vielen verschiedenen Faktoren, wie der Sprache, den verschiedenen Inhalten der Lehrpläne, der Lehrkultur und den unterschiedlichen Arten und Weisen, wie die Interaktion zwischen Lernenden und Lehrenden ist. Es kann vermutet werden, dass in 25 Jahren überall auf der Erde Internet verfügbar sein. Damit ist es möglich, jedem Lernenden AIED Systeme zur Verfügung gestellt werden kann. Es muss daher in dem Bereich interkulturelle Lehre geforscht werden, damit AIED Systeme entwicklet werden können, die allen Lernenden helfen. \cite[S. 9f]{Pinkwart.2016}

      \item \textbf{Practical impact (dt. praktische Auswirkungen)} \\
            Es gibt bereits einige AIED Systeme, die in Schulen eingesetzt werden. Dies passiert oft mit kommerziellen Initiativen innerhalb des AIED-Bereichs.
            Diese Verbindungen zu Schulen sind zwar wichtig, um den Einfluss von AIED Systemen auf die Bildung weiter zu erhöhen. Dennoch werden in Forschungssystemen gewisse Interessensgruppen, wie etwa Lehrkäfte und Schulen, oft nicht eingebunden.
            Besonders in Schulen ist die Frage nach technischem Support, Fehlerbehebungen, Stabilität und einer kontinuierlichen Verfügbarkeit sehr wichtig und wird häufiger nicht ausreichend beachtet.
            Daher besteht die Herausforderung darin Wege zu finden, die über Forschungsysteme hinausgehen und dauerhaft bestehen. \cite[S. 10]{Pinkwart.2016}

      \item \textbf{Privacy (dt. Privatspähre)} \\
            Dadurch das AIED Systeme als Begeleiter fürs lebenslange Lernen dienen sollen, werden viele Daten benötigt und gesammelt \cite[S. 10]{Pinkwart.2016}.
            Diese müssen geschützt werden, damit sie z. B. nicht genutzt werden, um Kanidaten bei einer Bewerbung im Vorhinein auzusortieren, wenn die Daten von Firmen verkauft werden \cite[S. 9f]{Pinkwart.2016}.
            Das Risiko entsteht, wenn AIED Systeme von privaten Unternehmen entwickelt werden. Daher spielt der Datenschutz eine sehr wichtige Rolle. Die Regeln zur Einhaltung des Datenschutzes könnten sich in Zukunft verschärfen.
            Dies ist zwar für Benutzende vorteilhaft, stellt jedoch für die Unternehmen eine weitere Herausforderung dar. Unternehmen sind meist gewinnorientiert und erzielen heutzutage mit den Daten ihrer Nutzer Gewinn.
            Ist dies durch die strikteren Regeln nicht mehr möglich, muss eine andere Möglichkeit gesucht werden. Eine der Möglichkeiten wäre, Werbung zu schalten, was jedoch den Lernfluss stören könnte.
            Eine andere Möglichkeit wäre eine Finanzierung durch Lizenzierung. Das könnte jedoch zu dem Problem führen, dass sich einige Institutionen sich diese Software nicht leisten können.
            Sollte also mit den Daten der Lernenden Gewinn erzielt werden, werden klare und für den Nutzer transparente Regeln nötig. \cite[S. 10f]{Pinkwart.2016}

      \item \textbf{Interaction methods (dt. Interaktionsmethoden)} \\
            Die vierte Herausforderung sind die Interaktionsmethoden. Durch die Entwicklung von Mensch-Computer-Interaktion ist die Interaktion der Lernenden und den AIED Systemen einfacher und intuitiver als früher.
            Daher ist es möglich, dass sich dieser Trend weiter fortsetzt. Es muss überlegt werden, wie die Lernenden in Zukunft mit AIED Systemen interagieren sollen.
            Es muss für die Lernednen möglich sein, leicht Informationen einzugeben und bequemes Feedback zurückzubekommen.
            Die Schwierigkeit besteht darin, die Eingaben, die über unterschiedliche Kanäle (z. B. Sprache oder Text) eingehen, effektiv und fehlerfrei zu empfangen, damit diese analysiert und genutzt werden können. \cite[S. 11]{Pinkwart.2016}

      \item \textbf{Collaboration at scale (dt. Zusammenarbeit im großen Maßstab)} \\
            Es gibt viele globale online Kurse, die Tausende von Lernenden erreichen. Diese setzen jedoch nur auf einfache Bildungstechnologien wie Video, Texte, Multiple-Choice-Tests und Diskussionsforen.
            Möglichkeiten und einige Grundlagen für das globale gemeinsame Lernen mit intelligenten Techniken, wie eine personalisierte Anleitung und die Möglichkeit für die Zusammenarbeit, sind schon vorhanden.
            Damit Lehrkräfte und Teilnehmer an großen Online-Kursen besser unterstützt und gefördert werden, muss an diesen Techniken noch weiter geforscht werden. \cite[S. 11]{Pinkwart.2016}

      \item \textbf{Effectiveness in multiple domains (dt. Effektivität in verschiedenen Bereichen)} \\
            Die sechste Herausforderung besteht darin, dass AIED Systeme auch in anderen Bereich aktiv eingesetzt werden, die heutzutage nicht vertreten sind. Solche Bereiche sind die, die nicht genau definiert sind.
            Heutzutage gibt es viele AIED Systemen in Bereichen, wo die Domäne bekannt und gut modeliert werden kann z. B. Geometrie und Mathematik.
            Damit in Zukunft auch Ethikaufsätze oder ein Virtual-Reality-Chirurgietraining mithilfe von AIED Systemen abgedeckt werden können, wird weiter Forschung und Entwicklung in  wissenschaftlichen Bereichen, wie der Technologie zur Verarbeitung natürlicher Sprache oder der Wissensdarstellung benötigt. \cite[S. 11f]{Pinkwart.2016}


      \item \textbf{Role of ac{AIED} in educational technology (dt. Rolle von \ac{AIED} in Bildungstechnologien)} \\
            In vielen Bereichen, wie z. B. der Mensch-Computer-Interaktion, gibt es bereits globale Konferenzen (CHI-Konferenz) und Veranstaltungen, die einen gewissen Prestige vorweisen und die weitere Entwicklung des Bereichs stark beeinflussen können.
            Im Bereich der Bildungtechnologien gibt es derzeitig nichts Vergleichbares.
            Es gibt jedoch eine große Bandbreite an Veranstaltungen und damit viele Richtungen, in welche die Forschung gehen kann.
            Dies ist erstmal nichts Negatives, da sich so ein Bereich auch weiterentwickelt.
            Die Herausforderung besteht darin, eine globale Gemeinschaft der Forschung im Bereich der Bildungstechnologie aufzubauen, um so starke Fragmentierung zu vermeiden.
            Eine zu starke Fragmentierung kann dazu führen, das sich ein Bereich nicht stark und gut weiterentwickelt und damit keine Gesamtwirkung nachweisen kann. \cite[S. 12]{Pinkwart.2016}

\end{enumerate}