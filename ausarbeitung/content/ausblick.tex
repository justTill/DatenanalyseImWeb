\chapter{Ausblick} 
Der Aublick beschäftigt sich mit den Fragen, welche aufbauenden arbeiten können in Zukunft erwartet werden und gibt es Aufgabenstellungen, im Umfang einer Masterarbeit, die identifiziert werden können.
\\ \noindent
Da es nicht viele Lernangebote über KI im deutschen Sprachraum gibt, kann man in Zukunft erwarten, dass es dort eine größere Auswahl an verschiednenen Lerntools und Lernportalen geben wird.
In diesen Portalen wird es vermutlich hauptsächlich um KI gehen und weniger den speziellen kontext der KI in der Lere - AIED -. Daher könnte man in einer Masterarbeit neue Lerninhalte erstellen und veräffentlichen.
Im Rahmen einer Masterarbeit kann auch die Vorhersage, ob Studierende den Kurs bestehen oder durchfallen, an der HSD mithilfe von alten Daten getestet werden.
Dort sollten genauere Zugriffsdaten und Zwischenergebnisse von Aufgaben mit einbezogen werden. Dies ist bei einem Test an der HSD besonders wichtig, da dort die Kurslänge in einem Sommersemester ungefähr 18 Wochen beträgt.
Im Anschluss und im Rahmen einer weiteren Masterarbeit könnte die KI mit den Ergebnissen verbessert und mit dem Einverständnis der Studierenden im laufenden Betrieb getestet werden.
