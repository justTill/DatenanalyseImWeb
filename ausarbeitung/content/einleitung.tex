\chapter{Einleitung}
Das Thema Künstliche Intelligenz erlangt zunehmend an Relevanz und durchdringt viele Felder. Auch die Hochschullehre kann vom Einsatz künstlicher Intelligenzen profitieren. Es existieren erste Versuche intelligente Systeme in die Lehre zu integrieren, wie etwa Chatbots in eingesetzten Lernmanagementsystemen, qualitative Auswertungen von Lernendendaten durch Learning Analytics und weitere \cite*[S. 18, S. 14ff.]{Witt.2020}.

Zugriffsdaten sind Daten, die beim Zugriff auf (Web-)Ressourcen erzeugt werden. Sie können bspw. für das Reporoduzieren und Beheben von Fehlern eingesetzt werden. Sie können jedoch auch für Analysezwecke genutzt werden, um vom Zugriffsverhalten Rückschlüsse auf andere Erkenntnisse zu erlangen.

Für die Entwicklung von KIs sind Daten notwendig, welche für das Training einer der KI eingesetzt werden. Die zentrale Arbeitsfrage dieser Ausarbeitung ist, wie Zugriffsdaten für den Einsatz von künstlichen Intelligenzen im Kontext der Hochschullehre eingesetzt werden können.

Hierfür wird zunächst eine Einführung in das Themenfeld KI im Kontext der Hochschullehre gegeben. Die Paper \glqq Künstliche Intelligenz in der Hochschullehre\grqq{} von de Witt, Rampelt und Pinkwart und \glqq Another 25 Years of AIED? Challenges and Opportunities for Intelligent Educational Technologies of the Future\grqq{} von Pinkwart geben einen Überblick über aktuelle Entwicklungen in der Hochschullehre mit KIs.

Aufgabenstellung beschreiben und in DAW einordnen
\\
\noindent
Historische Entwicklung. Gibt es erst seit 25 Jahren, ist also noch in den Kinderschuhen... es wird viel geforscht
Diagramm finden warum das Thema Wichtig ist, wie viele Hochschulen setzten KIs ein im verlaufe der Jahren ($https://hochschulforumdigitalisierung.de/sites/default/files/dateien/HFD_AP_59_Kuenstliche_Intelligenz_Hochschulen_HIS-HE.pdf Seite 9)$
\\
\noindent
Als erstes ........