\chapter{Fazit}
KI wird viele Bereiche durchdringen, darunter sehr wahrscheinlich auch die (Hochschul) Lehre.
Es gibt bereits Studien, die auf einen Anstieg der Nachfrage nach KI Fachkräften deuten \cite[S. 26]{Witt.2020}.
KI kann für die Verbesserung der Lehre eingesetzt werden. Davon können Lernende, Lehrende und auch gesamte Organisationen im Rahmen von Learning Analytics profitieren.
Gründe dafür sind, dass bspw. Daten von Lernenden analysiert und auf deren Grundlage Empfehlungen für die Lernstrategie der Lernenden getätigt werden können. %\cite[S. 14ff]{Witt.2020}
Intelligente Tutor- und Lernsysteme können für die Verbesserung von Lernergebnissen durch automatisierte Betreuung eingesetzt werden. %\cite[S. 24]{Witt.2020}.
Lehrende können durch Einsatz von KI ihre Didaktik und die Qualität ihrer Inhalte verbessern.
Auch können sie durch automatisierte Auswertungen von Einreichungen entlastet werden \cite[S. 14ff]{Witt.2020}
\\
\\ \noindent
Es existiert noch eine Vielzahl an Herausforderungen, um KI in der Lehre effektiv einsetzen zu können.
Bei der Entwicklung von KIs müssen demnach soziale und kulturelle Hintergründe berücksichtigt werden.
Menschen sind verschieden und haben unterschiedliche persönliche Eigenschaften, die einen Einfluss auf ihren Lernprozess haben. \cite[S. 9ff]{Witt.2020}
KI und Ethik sind ein wichtiges Feld, die bei der Entwicklung, der Auswahl und beim Einsatz von KI Anwendungen einen Einfluss haben.
Ein Aspekt ist die Qualität der Daten. Damit eine KI nicht rassistische, diskriminierende und anderweitig vorurteilsbehaftete Entscheidungen trifft, ist die Qualität der Daten zu sichern.
Es muss darauf geachtet werden, dass die Trainingsdaten einer KI frei von Bias ist, damit keine Personengruppen durch die KI benachteiligt werden. \cite[S. 39]{Witt.2020} 
Weiterhin ist die Transparenz im Rahmen von Explainable AI relevant. Dadurch werden Verwendungszwecke von (personenbezogenen) Daten und die Erklärung von Auswertungen, wie bspw. Empfehlungen, nachvollziehbarer. \cite[S.10f; S. 39ff;]{Witt.2020}
Außerdem sollte es klare Regeln geben, wie mit den (personenbezogenen) Daten der Benutzer umgegangen wird.
Es sollte es unter keinen Umständen möglich sein, dass die Daten auf andere Art verwendet werden können, also Feedback zum Benutzter geben zu können oder die KI zu verbessern.
Falls Daten im kommerziellen Sinne verkauft werden, soll es nicht möglich sein, mithilfe der Daten die Identitäten der Benutzer herausfinden zu können.
Somit kann verhindert werden, dass die Daten benutzt werden, um z. B. Bewerberkandidaten im vorhinein schon auszufiltern \cite[S. 8f]{Pinkwart.2016}. 
Die Interaktion zwischen Menschen und Maschinen hat sich gewandelt und der Trend führt sich fort. Es gilt zu überlegen, wie in Zukunft die Interaktion aussieht.
Für KI in der Lehre ist bei der Interaktion wichtig, dass Informationen leicht einzugeben und Feedback von der KI bequem einsehbar sind. 
Des Weiteren muss viel an der Effektivität von KIs in verschiedenen Bereichen der Lehre geforscht werden, KIs sollen Lernenden nicht nur in einem Fach oder Kurs unterstützen sondern ein ganzen Leben lang.
Die KIs müssen aber nicht nur die Bedürfnisse der Lernenden, sondern auch der Lehrenden und anderen Bildungsorganisationen erfüllen.
Ein Beispiel wäre die Frage nach dem Technischen Support bzw. Fehlerbehebung, die während des laufenden Betriebes auftreten können.\cite[S. 10ff]{Pinkwart.2016}
\\
\\ \noindent
Eine KI, welche auf Basis von Zugriffsdaten auf der Lernplattform Moodle arbeitet, kann vorherhsagen, ob Studierende einen Kurs wahrscheinlich bestehen oder nicht.
Damit die KI zu einen früheren Zeitpunkt studierende ermitteln kann, die voraussichtlich den Kurs nicht bestehen muss diese Verbessert werden.
Eine Verbesserung kann erreicht werden, wenn mehr Zugriffsdaten und auch andere Daten, wie Zwischenergebnisse von Aufgaben, von der KI mit in die Entscheidungsfindung einbezogen werden. \cite[S. 14f]{Quinn.2020}
\\ \noindent
\\ \noindent
Auch wenn die KI, wie in dem Paper \glqq Prediction of student academic performance using Moodle data from a Further Education setting\grqq{} nicht alle Studierenden ermitteln kann, die möglichweise durchfallen \cite[S. 16]{Quinn.2020}.
Ist jedoch möglich, einigen der Studierenden zu Hilfe anzubieten. Es muss jedoch bedachtet werden, dass es auch andere Gründe gegeben kann durch einen Kurs zu fallen, als die Inhalte nicht zu verstehen.
Es kann sehr gut vorkommen, dass Studierende aus externen Gründen, z. B. familiären, es nicht schaffen sich auf den Kurs vorzubereiten.
Dort wären dann die Hilfestellungen, die der Dozent anbieten kann, nicht zielführend.
Daher sollten nur Vorhersagen bei Studierenden getroffen werden, die ein explizites Einverständnis gegeben haben.
