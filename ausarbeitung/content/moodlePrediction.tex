\begin{comment}
Viele Systeme erzeugen während sie benutzt werden Daten, auch genannt Zugriffsdaten.
Diese Daten werden gespeichert und genutzt um Informationen über die Benutzer zu gelangen. 
Solche Informationen können benutzt werden um neue Produkte zu erschaffen oder zu verbessern.
Zurgiffsdaten werden auch von Systemen, die in der Bildung eingesetzt werden gesammelt z. B Lernmanagementsysteme - oder auch virtuelle Lernumgebungen oder Kursmanagemensysteme genannt - wie Moodle.
Die Aufgabe von Bildungsintitutionen ist es, die Daten zu Nutzen, um das Lehren und lernen zu verbessern.
\end{comment}
Rory Quin und Professor Geraldine Gray sind an der \glqq Technological University Dublin\grqq{} im Fachbereich für \glqq Informatics and Engineering
PhD \grqq{} angestelt. Sie haben in dem Artikel \glqq Prediction of student academic performance using Moodle data from a Further Education setting\grqq{} welches im Irish Journal of Technology Enhanced Learning im jahre 2019 veröffentlich wurde und nach researchgate 14 mal Zitiert worden ist, untersucht ob es Möglich ist anhand von Zugriffsdaten des Lernmanagementsystems Moodle die an einer Hochschule gesammelt wurden,
vorhersagen über die Zukünftigen Laufbahn eines Studentens innerhalb eines Kurses zu treffen. Diese Art von Forschung wurde schon von einigen anderen Professoren und Wissenschaftlern untersucht, jedoch immer nur an Universitäten und nie mit Daten von einer Hochschule.
\\ \noindent
Rory Quin und Professor Geraldine Gray versuchen in dem Paper die Frage, ob es möglich ist, mit Aktivitätsdaten von Moodle, die während der dauer eines Kurses gesammelt werden vorherzusagen, wie gut ein Student in dem Kurs abschneidet und damit auch ob dieser den Kurs besteht oder durchfällt zu beantworten. 
Ebenso versuchen Sie die Frage, ob es möglich ist, mit den selben Daten jedoch nur von den ersten sechs bzw. zehn Wochen vorherzusagen ob ein Student den Kurs bestehen oder durchfallen wird.
Dies hat den Grund, dass Dozenten Präventionen einleiten können für Studenten die Gefahr laufen den Kurs nicht zu bestehen. Dies würde nicht nur den Studenten helfen sondern auch der Hochschule und dem Dozenten, denn diese würde langfristig den Zeit und Geld sparen.

\subsection{Vorgehen}
Um die Fragen zu beantworten werden Daten von insgesamt 29 Kursen von 9 verschiedenen Modulen im selben Fachbereich und unter der Leitung eines Dozentens genommen.
Die Kurse fanden zwischen den Jahren 2011 und 2018 statt. Die Kurse hatten unterschiedlich lange Laufzeiten, der Kürzetes Kurs hat eine Länge von 11 Wochen und der Längste 33 Wochen.
Insgesamt konnten so Daten von 690 Kursteilnehmern. Die 690 Kursteilnehmer setzten sich aus insgesamt 410 Studierenden zusammen. 
\\
Es werden folgende Zugriffsdaten verwendet, um vorhersagen zu treffen.
LISTE VON ZUGRIFFSDATEN /BILDER EINIGE NENNEN
\\
\noindent
Die erste Klassifizierungsaufgabe besteht darin den Notenbereich der Studierenden hervorzusagen und damit ob diese den Kurs bestehen oder durchfallen. Es gibt folgende Klassen.
\begin{enumerate}
    \item Early Exit (Abbruch)
    \item Fail (Durchgefallen)
    \item Pass (Bestanden mit 50\% bis 64\%)
    \item Merit (Bestanden mit 65\% bis 79\%)
    \item Distinction (Bestanden mit 80\% bis 100\%)
\end{enumerate}
Um die zweite Frage zu beantworten werden die ersten beiden Klassen Early Exit und Fail, sowie die anderen drei Klassen zusammengefasst.
Es gibt jetzt nur noch die Klassen Bestanden und Durchgefallen.
\\
\noindent
\\
\noindent
Es werden vier verschiedene Algorithmen getestet, um die bestmögliche Vorhersagen zu treffen. Der Random Forest, Gradient Boosting, k Nearest
Neighbours und Linear Discriminant Analysis. Es werden sich für die Algorihmen getestet, da diese in anderen Forschungen bereits gute Ergebnisse erzielt.
Jeder der Algrotihmen wird mit 70\% der Daten trainiert. Die restlichen 30\% werden benutzt um die Genauigkeit zu überprüfen.
Die Genauigkeit wird außerdem mit der\glqq no-information rate\grqq{} verglichen. Diese gibt die Genauigkeit an, wenn immer die am häufigsten vorkommende Klasse vorhergesagt wird. 

\subsection{Ergebnis}
Das Trainieren und Testen mit allen Daten ergibt, dass alle Algorithmen einen ähnliche Genauigkeit haben, wenn es darum geht den Notenbreich vorherzusagen.
Sie unterscheiden sich nur in wenigen Prozentpunkten. Der Random Forest Algorihmus trifft mit einer Genauigkeit von 60,5\% Genauigkeit die besten vorhersagen.
Wenn es nur darum geht vorherzusagen, ob ein Student den Kurs besteht oder durchfällt hat der Random Forest Algorithmus eine Genauigkeit von 92.2\% erreicht. 
Ach hier schneidet der Algrithmus am besten ab. Mit den 92,2\% ist der Algorihmus Signifikant besser als die \glqq no information rate\grqq{}. 
Diese liegt aufgrund der Tatsache, das es mehr Studierende gab die Kurse bestanden haben, als die die nicht bestanden haben bei 73.5\%. 
Der Random Forest Algorihmus kann insgesammt 121 von 152 Studierende die den Kurs bestanden haben richtig hervorsagen und 42 von 54 Studierende die den Kurs nicht bestanden haben.

\subsubsection{Frage 2}
Das Trainieren und Testen mit Daten von sechs Wochen ergibt, dass kein Algorihmus bessere voraussagen treffen kann, als wenn man die \glqq no information rate\grqq{} benutzt,
die bei 75,5\% liegt. Die unterschiedliche \glqq no information rate\grqq{}, setzt sich daraus zusammen, dass noch nicht alle Studierenden innerhalb der ersten 6 Wochen des Kurses Moodle benutzen.
\\ \noindent \\ \noindent
Wurden hingegen die Daten der ersten zehn Wochen benutzt kann eine Signifikante verbesserung gegenüber der \glqq no information rate\grqq{} erreicht werden. 
Auch hier schneidet der Random Forest Algorithmus am besten ab. Er erreichte eine Genauigkeit von 82.18\% wenn es darum ging zu bestimmen ob studierende den Kurs bestehn oder nict.
Die \glqq no information rate\grqq{} liegt bei 74.8\%.
\\ \noindent \newline \noindent
\\ \noindent \newline \noindent
Am meisten Einfluss auf die Vorhersage, wenn alle Daten oder die der ersten 10 Wochen benutzt wurden, hatte die Anzahl an Tagen an denen sich die Studierenden in den Kurs eingeloggt haben.

\subsection{Fazit}
Die Testergebnisse zeigen, dass es möglich ist mihilfe von Zugriffsdaten die von Moodle im laufe eines Kurses gesammelt werden Signifikant bessere vorhersagen über die Noten der Studierenden zu treffen als mit der \glqq no-information rate\grqq{}.
Dabei haben alle Algorithmen gut abgeschlossen und sich nur innerhalb von maximal fünf Prozentpunkten unterschieden. Der Random Forest Algorithmus schließt dabei mit 60.5\% am ab.
Der Random Forest Algorithmus kann 97\% der Studierenden die den Kurs bestehen und 78\% der Studierenden die den Kurs nicht bestehen vohersagen.
\\ \noindent \\ \noindent
Es Reicht jedoch nicht aus am Ende eines Kurses voherzusagen zu können ob ein Studierender den Kurs besteht oder nicht.
Ein frühzeitiger Hinweise welche Studierenden möglicherweise nicht bestehen ,wäre für die Lehrenden Hilfreich, da sie so gezielt Hilfe leisten können.
Das Testergebnis zeigt, dass es nicht möglich ist mit Daten von sechs Wochen festzustellen welche Studierenden möglicherweise Durchfallen. 
Mit den gesammelten Daten von zehn Wochen kann jedoch eine Signifikant bessere Vorhersage getroffen werden als mit der \glqq no-information rate\grqq{}.
Trotzdem können nur weniger als die Häflte der Studierenden ermittelt werden, die Durchfallen. Dies deutet darauf hin, dass die Zugriffsdaten von Moodle hilfreich für ein Frühwarnsystem sein können.
Mehr Daten wären hilfreich, wenn es darum geht bessere Vorhersagen treffen zu können. Man könnte noch feiner Zugriffsdaten, Ergebnise von Zwischenabgaben, oder Anzahl an Interaktionen mit anderen Studierenden einzubeziehen.
\\ \noindent \\ \noindent
Das der Reguläre Login so eine hohe relevanz bei den Vorhersagen hat, zeigt darauf hin, dass diese Student ein besseres Lern und Zeit Management haben. Dies ist ein wichtiger Faktor für den Erfolg.
\\ \noindent \\ \noindent
Abschließend kann gesagt werden, die Zugriffsdaten von Moodle können benutzt werden um ein Frühwarnsystem zu erstellen. Mehr und genauere Daten wären hilffreicher.
Die Ergebnisse dieser Studie sollten jedoch noch mit einer Größeren Datenmenge wiederholt werden um die Ergebnisse zu bestätigen.

\begin{comment}
\section{Notizen}
Um die Fragen zu beantworten wurden Daten von 29 verschiedenen Kursen zu 9 verschiedenen modulen zwischen den Jahren 2011 und 2018 ausgewählt.
Alle Module kommen aus dem Selben Fachbereich und dem selben Dozenten um eine homogene Maßen an Daten anzusammeln.
Insgesamt konnten so Kursdaten von 690 Studenten (410 waren Indivuell, rest selber Student) gesammelt werden, wobei 83 davon, den Kurs frühzeitig verlassen haben.

Aufgrund des doch Speziellen Datensatzes können die Ergebnisse nur im Context des kleines Stundenten kreises gesehen werden.



\section{Klassifizierungen}
Um die Erste Frage zu beanrtworten wurde eine KI entwickelt. Diese KI kann insgesamt 5 verschiedene Vorherssagen treffen.
"Distinction" (80\%+)
"Merit" 65\%- 79\%)
"Pass" (50\% - 64\%)
"Fail" (1\% bis 49\%)
"Early Exit"(0\% oder keine Note)

Um die Zweite Frage zu beanrtworten wurden die ersten 3 Klassen zur Klasse "Pass" zusammengefasst und die letzten beiden zu "Fail"

\section{Daten}
Es wurden nur Zugriffsdaten von Studenten zum Trainieren und Validieren der Ki genommen, die von Dozenten mussten entfernt werden
Es wurden folgende Zugriffsdaten verwendet: LISTE im PAPER!!!

\section{Algrothmen}
Genauer Beschreiben ?
Random Forest:
Graduent Boosting:
K nearest Neigbours:
Linear Discriminat Analysis:

Dies sind verschiedene Algrothmen die benutzt werden können um Daten in Klassen zu Ordnen.

\section{Model Building an Evaluation}
Data-Split in 70-30

Kappa Statistic gib an wie gut die vom Modell vorausgesagten Klassen mit den richtigen Klassen überinstimen, zur Kontrolle wird ein  
zu einem Zuflligen Klassifiziere, der anhand der Häufigkeit der vorkommen von Klassen rät um welche Klasse es sich handelt genommen
0.02 = Fair
021-040. Moderat
041-060 Zustimmend
061-080 wesentiche Zustimmung
081-10 Perfekte ZUstimmung

no-information Rate = genauigkeit wenn immer die Klasse mit den meisten laben in dem Testset gerraten wird

\section{Ergebnisse}
Das trainierungd und Testen der Ki hat ergeben, das alle Algorithmen eine sehr ähnliche Genuigkeit haben.
Die Genaurigkeit liegt bei dem Model zu 95\% in dem Bereich 53\% bis 67\%. Des Weiteren waren Alle
Algorithmen besser als die no-information Rate. Am besten konnte vorrausgesagt werden, wenn ein Student besonders gut den Kurs besteht oder den Kurs frühzeitig beendet
Alles andere konnte nicht gut predicted werden.

Wenn anhand von der Zugfiffsdaten für die gesamte Länge des Kurses, konnte, wenn nur festegstellt werden soll
ob ein Student besteht oder nicht, mithilfe des Random Forest Algrithmus wurde eine Genauigkeit von 92\% erreicht
(148 von 152 Studenten die bestanden haben und 42 von 54 die nicht bestanden haben konnten korrekt vorhergesagt werden)

\subsection{Ergebnisse mit 6 Wochen Daten und 10 Wochen Daten}
Wenn die Ki nur mit 6Wochen alten Daten voraussagen treffen soll,
hat diese nicht besser abgeschnitten als die no-information rate.
Wurden jedoch 10 Wochen alte Ergebnisse genommen, konnte die deutlich bessere
vorhersagen treffen als die no-information rate. Auch hier waren alle Algorthmen besser,
am besten war jedoch dedr Random Forest Algorithmus mit 82.18\%

\section{Korrelation der Variablen}

Pass/Fail all Data:
Regular: und Assignment Submitted


Random Forest /Pass/Fail 10 Weeks
Regular / Assignment Vies
Course Views
pm.Early
am.late


\section{Disskusion}
Frage 1: Wenn alle Daten benutzt werden ist es möglich die Notesignificant besser zu bestimmen als mit der no-information-rate 
am besten war dabeider Random Forest mit einer genaurugkeit5 von 60,5. Dies deutet darauf hin, das die Moodle aktivität benutzt werden kann um die Noten vorherzusagen.
Dabei war die Unterschiedung, ob ein Student sehr gut besteht oder abbricht am besten zu bestimmen, das kann daran liegen, das dort das Nutezrverhalten extrem eindeutig ist.
Wenn man nur bestimmen möchte ob die Studenten bestehen oder nicht können die Daten sehr gut genutzt werden. Dort hat man eine Genauitgkeit von 92.2\%
Dieses Ergebnis stimmt auch mit vielen anderen Studien überrein.

Das Gute Ergebniss kann daran liegen, das durch den längeren Zeitraum mehr Datevrohanden waren. Im Gegensatz zu anderen Studien waren hier weniger als 83 Prizenzt der Daten passiv sonst sind das oft über 95 Prozent.



Frage 2: Für den Kurzen Zeitraume von 6Woche  hat keiner einen Algrotihmen eine Verbesserung gebracht. Mit Daten von 10 Wochen hat die PErfomance etwas verbessert.
Jedoch auch hier kann nur weniger als die Hälfte der Failing Student vorhergesagt werden. Dies deutet darauf hin, das LMS Zugfiffsdaten hilfereich für ein frühwarnsystem sein können,
Jedoch werden noch mehr Daten benötigt z. B. eine noch genauere LMS Nutzungdaten oder Assessment results, Studentische Interaktion mit anderen. Dies wurde auch in anderen Studien schon gezeigt, das diese Daten helfen können.

Trotz der hohen false-positiven kann es hilfreich sein so ein frühwarnsystem einzurichten, da dadruch trotzdem einige Studenten erreicht, die dadruch hilfe bekommen können
und das die Schulen und den Studenten weniger kostet, als wenn diese einfach am ende Durchfallen.


Es wurde gezeigt, das das Regulräre Login darauf hnweißt, das ein Student eher besteht. Viele Forscher meinen dies Zeigt darauf, das die Studenten ein besseres Lern/Zeit Managementhaben und dies ein wichtiger Faktor für den erfolg ist.



Um die Modelle zu verbessern könnten feinere Zugriffsdaten gesammelt werden, sowie Ergebniss von Aufgaben beachtet werden. Dozenten sollten Aufgaben früh ihn ihren Kurs einbauen und die Möglichkeiten von LMS nutzen um Studentische ineraktivität zu fördern.


\section{Ergebnis}
Moodle Data kann benutzt werden um vorherzusagen, wie gut ein Student besteht un ob dieser Besteht. Um dies jedoch Praktisch zu benutzen, währe ein Frühwarnsystem gut, Moodle daten reichen dazu nicht ausreichen aus. 
Die getestetn Modelle sollten daher mit mehr Daten, genaueren Zugriffsdaten und Aufgaben ergebnisse miteinbezogen werden. 
Des weitren waren alle Zugriffsdaten nur von Kursen die von einem Doztenten geleitet wurden. Daher ist das ergebnis nicht gut Generalisierbar
\end{comment}