
Das Paper \glqq{}Whitepaper künstliche Intelligenz in der Hochschule\grqq{} umfasst eine Sammlung von verschiedenen Werken zum Thema Künstliche Intelligenz in der Hochschullehre. Das Paper ist in die folgenden vier Kernthemen gegliedert:
\begin{enumerate}
    \item Lehren und Lernen mit KI in der Hochschulausbildung
    \item Lehren und Lernen über KI in der Hochschulausbildung
    \item KI und Ethik in der Hochschulausbildung
    \item Zukunftsperspektiven für KI in der Hochschulbildung
\end{enumerate}

Nachfolgend folgt eine inhaltliche Zusammenfassung der vier Kernthemen. Dabei wird auf die wichtigsten Erkenntnisse und Aussagen eingegangen.


\subsection{Lehren und Lernen mit KI in der Hochschulausbildung}
Beim Kapitel \glqq{}Lehren und Lernen mit KI in der Hochschulausbildung\grqq{} geht es darum, wie künstliche Intelligenzen eingesetzt werden können, um die Lehre sowohl aufseiten der Lernenden, als auch für die Lehrenden und Bildungsorganisationen zu verbessern. Es werden mehrere Gründe genannt, warum KI eingesetzt werden sollte:
\begin{itemize}
    \item Verwendung von Daten zur Unterstützung von Studierenden bei der Weiterentwicklung des eigenen Lernverhaltens und Lehrenden bei der Verbesserung ihrer Didaktik.
    \item Förderung von kritischem und kreativem Denken
    \item Entdeckung neuer Erkenntnisse durch Analyse von Daten
    \item Vorzeitige Erkennung von Risiko-Studierenden
\end{itemize}

Die Gründe werden davon gestützt, dass eine Reihe von Herausforderungen in der aktuellen Lehre bestehen. Als Beispiel wird genannt, dass die Zeit von Lehrenden begrenzt ist, sodass sie Lernende nicht immer vollständig betreuen können. Als mögliche Maßnahme werden virtuelle intelligente Assistenten aufgeführt, die bspw. Lernende anstelle von Lehrenden betreuen und Hilfestellungen leisten. Auch die Auswertung von längeren Texteinreichungen wird als aktuelle Herausforderungen herausgestellt. Hierbei werden Automated Essay Scoring Anwendungen vorgeschlagen, die solche Texteinreichungen systematisch analysieren, bewerten und Lernenden Feedback geben können.

\subsubsection*{Learning Analytics}
Ein aufgeführtes Konzept zur intelilgenten Analyse von Leistungsdaten von Lernenden ist Learning Analytics. Learning Analytics kann als Prozess verstanden werden, welcher den Prozess des Lernens unterstützt. Der Prozess wird in drei Ebenen eingeteilt: Mikroebene, Mesoeebene und Makroebene. Der Prozess kann je Ebene unterschiedliche Zielgruppen unterstützen, wie Lernende, Lehrende oder auch gesamte Organisationen. Die Ebenen beziehen sich dabei auf den Umfang und zeitlichen Kontext des Lernens. In der Mikroebene können Lernende bspw. im Rahmen einer Lernsitzung oder im Verlaufe eines Kurses unterstützt werden. In der Mesoebene können Lernende über ein gesamtes Semester begleitet und Daten kursübergreifend ausgewertet werden. Auf Grundlage dieser Daten kann Learning Analytics auch für die Erkennung von gefährdeten Lernenden eingesetzt werden.

Auch bei Learning Analytics bestehen Herausforderungen, die es zu addressieren gilt. Allen voran ist zu beachten, dass Learning Analytics bei der Auswertung lediglich einen Ausschnitt des Lernprozesses berücksichtigen kann. Weiterhin sollten intelligente Systeme die persönlichen Eigenschaften sowie den sozialen und kulturellen Kontext der Lernenden berücksichtigen.

\subsubsection*{Arten von Künstlichen Intelligenzen in der Hochschulbildung}
Das Whitepaper listet eine Reihe von Arten künstlicher Intelligenzen in Hinsicht auf die Hochschullehre auf.
\paragraph*{Personalisierte adaptive Lernumgebungen (ALU)}
Unter Personalisierten adaptiven Lernumgebungen werden digitale Lernumgebungen für die autonome Lehre gut strukturierter Inhalte verstanden. Dabei werden drei Modelle unterschieden: Domänenmodell (Fach), Lernermodell und Didaktikmodell. Das Domänenmodell behandelt den Inhalt des jeweiligen Faches. Das Lernermodell betrifft den Zustand des Lernenden und welches Wissen er besitzt. Das Didaktikmodell schlägt auf Basis der Erkenntnisse des Lernermodells Lernstrategien und Handlungsempfehlungen für Lernende vor.

\paragraph*{Chatbots}
Lernende können durch Chatbots moderiert und unterstützt werden, wobei sie gleichzeitig Lehrende entlasten. Als Beispiel wird ein Chatbot in einem Hochschulforum aufgeführt, welche Fragestellungen von Lernenden anstelle von Lehrenden beantwortet.

\paragraph*{Empfehlungssysteme}
Empfehlungssysteme sind bislang nur im Auswahlprozess eines Studiengangs im Einsatz.

\paragraph*{Edu-Robots}
Meist Begleitungen oder Ersatz für Lehrende in Kursen.

\paragraph*{KI-Schreibbots}
Generierung einzigartiger Texte im Hochschulkontext. Birgt die Herausforderung von erschwerter Überprüfbarkeit.

\subsubsection*{Veränderung der Betreuung}
Anstelle von klassischer Lehre fachlichen Wissens, sollen Lernende mehr zu kritischem Denken gefördert werden. Das schließt nicht aus, dass Fachkräfte für spezielle Fachrichtungen benötigt werden. Lernende sollen jedoch mehr dazu gebracht werden Wissen zu generieren, anstatt nur vermittelt zu bekommen. Daduch wird das Risiko eingegangen, dass nicht zu einer \glqq Allwissenheit\grqq{} ausgebildet wird. Es wird vermutet, dass sich die Betreuung von Lernenden durch Lehrende im Sinne eines Mentoring verändern wird. Beim Mentoring ist die Erkennung des Status des Lernenden wichtig, um darauf reaktiv handeln zu können. Das Mentoring wird in die Phasen Vorbereitung, Lernprozss und Nachbereitung aufgeteilt.
\begin{itemize}
    \item Vorbereitung: Erhebung der Lerninhalte und Planung der Ausgestaltung des Lernprozesses.
    \item Lernprozess: Ausführung der geplanten individualisierten Aktivitäten zur Aneignung der Lerninhalte.
    \item Nachbereitung: Reflexion des Lernprozesses über Lernerfolg.
\end{itemize}
Es wurde gezeigt, dass intelligente Tutor- und Lernsysteme zur Verbesserung der Lernergebnisse führen können. Es bestehen allerdings heute noch fehlende Tutoring- und Lernanwendungen mit KI. Bisher wruden nur Vorhersage von Erfolg und Misserfolg des Lernprozesses eingesetzt bzw. geprüft.

\subsection{Lehren und Lernen über KI in der Hochschulausbildung}
Damit KI effektiv eingesetzt werden kann, muss auch Wissen über KI vermittelt werden. Dies beginnt bei der Verwendung von KI und geht über Auswahl geeigneter KI bis hin zur Entwicklung neuer KI. Es wird erwartet, dass die Nachfrage an Fachkräften im Bereich KI stark wachsen wird. Daher sollte der Fokus mehr auf Lehre über KI gesetzt werden.

Damit die Lehre über KI stattfinden kann, müssen einige Herausforderungen beachtet und gelöst werden.

\paragraph*{Herausforderungen bei der Lehre über KI}
\begin{itemize}
    \item Uneinheitliches Verständnis über KI.
    \item Aktuell am meisten Verwendet in Ingenieurwissenschaften, Tendenz interdisziplinär.
    \item Fehlende Standards digitaler Lernangebote.
\end{itemize}

\subsubsection*{Digitale Lehre im Bereich KI}
Es wird gefordert, dass mehr qualifizierte digitale Lernangebote zu KI angeboten werden. Angebote sollten dabei für mehrere Zielgruppen zugänglich und verwendbar sein. Es sollen demnach auch Angebote vorhanden sein, die keine tiefen technischen Kenntnisse erfordern. Zudem sollten Lernangebote offen lizenziert und transparent sein.

\paragraph*{Deutsche Plattformen für digitale Lernangebote zum Thema KI}
\begin{itemize}
    \item oncampus (TH Lübeck)
    \item Hamburg Open Online University (HOUU)
    \item OPEN VHB (virtuelle Hochschule Bayern)
    \item OpenHPI (Potsdam)
    \item OER-Content.nrw (Nordrhein-Westfalen)
\end{itemize}
$\Rightarrow$ insgesamt sehr kleines Angebot an (kostenlosen) Kursen zum Thema KI.

\subsection{KI und Ethik in der Hochschulbildung}
KI und Ethik Motivation: Autonome Systeme können potenziell Menschen schaden. Daher: Anforderung an Transparenz und Verständnis über autnom getroffene Entscheidungen. Herausforderung in der Erklärbarkeit von KI Algorithmen (Forschung im Bereich \glqq Explainable AI\grqq{}).

\paragraph*{Qualität der Daten} Die Qualität der Daten beeinflusst unmittelbar die Ergebnisse der KI. Gefahr von Diskriminierung / Unterrepräsentation. Neben Algorithmen auch Daten qualitätssichern.

\paragraph*{Sensibilisierung für KI} Analyse der eigenen Daten und anschließende Auswertungen können Misstrauen durch Gefühle von Überwachung und Kontrolle erzeugen. Durch Analyse unterschiedlichster Daten wird zu Fremdbestimmung tendiert. $\rightarrow$ Sensibilierung schaffen um damit Akzeptanz und Verständnis zu erhöhen.

\subsection{Zukunftsperspektiven für KI in der Hochschulbildung}

\paragraph*{Sam als digitaler Assistent für Lernende}
Erzählung aus Sicht eines zukünftigen Studierenden. Sam als digitaler Assistent bietet folgende Funktionalitäten:
\begin{itemize}
    \item Einführung in Hochschule, Zugang zu Prüfungssystem, Lernmanagementsystem und Bibliothek
    \item Ansprechpartner für Fragen. Verweis auf externe (Studierendenberatung, Lehrende)
    \item Mentor für den eigenen Lernprozess unter Berücksichtigung der definierten Lernziele und -stils.
    \item Anpassung der Lerninhalte an eigenen Wissensstand
\end{itemize}

\paragraph*{Unterstützung von Lehrenden} Einsatz von KI zur Entlastung von Lehrenden. Dabei übernimmt KI folgende Funktionen:
\begin{itemize}
    \item Insgesamt Entlastung von Lehrenden.
    \item Automatisierte Überprüfung von Aufgaben.
    \item Feedback zur Verbesserung der Didaktik und Lerninhalte des Lehrenden.
\end{itemize}
Kurzfristig: Vermittlung von Wissen über Systeme an Lehrende. Mittelfristig: Betreuung der Studierenden beim Lernprozess zu kritischem Denken.