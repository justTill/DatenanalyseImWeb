
Das Paper \glqq{}Whitepaper künstliche Intelligenz in der Hochschule\grqq{} umfasst eine Sammlung von verschiedenen Werken zum Thema \ac{KI} in der Hochschullehre. Das Paper wurde über die Lernplattform KI-Campus im Jahre 2020 publiziert. Es ist von der Bildungswissenschaftlerin und Professorin für Bildungstheorie und Medienpädagogik Prof. Dr. habil. Claudia de Witt, dem Geschäftsstellenleiter des KI-Campus Florian Rampelt und dem Vizepräsidenten für Lehre und Studium der HU Berlin Prof. Dr. Niels Pinkwart verfasst. Prof. Dr. Claudia de Witt ist Leiterin von vielen Forschungsprojekten im Bereich von Künstlicher Intelligenz und Lehre, sowie Learning Analytics. \footnote{\url{https://www.fernuni-hagen.de/bildungswissenschaft/bildung-medien/team/claudia.dewitt.shtml}, zuletzt aufgerufen am 11.07.2022} Prof. Dr. Niels Pinkwart hat über 250 Publikationen und Forschungsprojekte, darunter auch welche, die sich mit der Anwendung von informationstechnologischen Methoden für Lehre und Studium befasst haben. \footnote{\url{https://link.springer.com/article/10.1007/s40593-016-0099-7}, zuletzt aufgerufen am
      10.07.2022}
\\ \\ \noindent
Das Paper ist in die folgenden vier Kernthemen gegliedert:
\begin{enumerate}
      \item Lehren und Lernen mit KI in der Hochschulausbildung
      \item Lehren und Lernen über KI in der Hochschulausbildung
      \item KI und Ethik in der Hochschulausbildung
      \item Zukunftsperspektiven für KI in der Hochschulbildung
\end{enumerate}
\noindent
Nachfolgend folgt eine inhaltliche Zusammenfassung der ersten drei Kernthemen. Dabei wird auf die wichtigsten Erkenntnisse und Aussagen eingegangen.

\subsection{Lehren und Lernen mit KI in der Hochschulausbildung}
Das Kapitel \glqq{}Lehren und Lernen mit KI in der Hochschulausbildung\grqq{} beschäftigt sich mit dem Thema, wie \ac{KI} eingesetzt werden kann, um die Lehre sowohl aufseiten der Lernenden, als auch für die Lehrenden und Bildungsorganisationen zu verbessern. Es gibt verschiedenen Gründe warum \ac{KI} eingesetzt werden soll: \cite*[S. 11f]{Witt.2020}
\begin{itemize}
      \item Verwendung von Daten zur Unterstützung von Studierenden bei der Weiterentwicklung des eigenen Lernverhaltens und Lehrenden bei der Verbesserung ihrer Didaktik. \cite*[S. 11]{Witt.2020}
      \item Förderung von kritischem und kreativem Denken \cite*[S. 11]{Witt.2020}
      \item Entdeckung neuer Erkenntnisse durch Analyse von Daten \cite*[S. 11]{Witt.2020}
      \item Vorzeitige Erkennung von Risiko-Studierenden \cite*[S. 11]{Witt.2020}
\end{itemize}
\noindent
Die Gründe werden davon gestützt, dass eine Reihe von Herausforderungen in der aktuellen Lehre bestehen.
Die Zeit von Lehrenden ist begrenzt, sodass sie Lernende nicht immer vollständig betreuen können.
Als mögliche Maßnahme können virtuelle intelligente Assistenten benutzt werden, die bspw. Lernende anstelle von Lehrenden betreuen und Hilfestellungen leisten.
Auch die Auswertung von längeren Texteinreichungen ist eine aktuelle Herausforderung.
Mögliche Lösung wären Automated Essay Scoring Anwendungen, die solche Texteinreichungen systematisch analysieren, bewerten und Lernenden Feedback geben können. \cite*[S. 12]{Witt.2020}

\subsubsection*{Learning Analytics}
Ein Konzept zur intelligenten Analyse von Leistungsdaten von Lernenden ist Learning Analytics.
Learning Analytics kann als Prozess verstanden werden, welcher den Prozess des Lernens unterstützt.
Der Prozess wird in drei Ebenen eingeteilt: Mikroebene, Mesoebene und Makroebene.
Der Prozess kann je Ebene unterschiedliche Zielgruppen unterstützen, wie Lernende, Lehrende oder auch gesamte Organisationen.
Die Ebenen beziehen sich dabei auf den Umfang und zeitlichen Kontext des Lernens.
In der Mikroebene können Lernende bspw. im Rahmen einer Lernsitzung oder im Verlaufe eines Kurses unterstützt werden.
In der Mesoebene können Lernende über ein gesamtes Semester begleitet und Daten kursübergreifend ausgewertet werden.
Auf Grundlage dieser Daten kann Learning Analytics auch für die Erkennung von gefährdeten Lernenden eingesetzt werden. \cite*[S. 14ff]{Witt.2020}
\\ \noindent
Auch bei Learning Analytics bestehen Herausforderungen, die es zu addressieren gilt.
Allen voran ist zu beachten, dass Learning Analytics bei der Auswertung lediglich einen Ausschnitt des Lernprozesses berücksichtigen kann.
Weiterhin sollten intelligente Systeme die persönlichen Eigenschaften, sowie den sozialen und kulturellen Kontext der Lernenden berücksichtigen. \cite*[S. 16]{Witt.2020}
\\ \\ \noindent
Es gibt verschiedene Arten von KIs in der Hochschulbildung, von denen einige im Folgenden beschrieben werden.

\paragraph*{Personalisierte adaptive Lernumgebungen (ALU)}
Unter Personalisierten adaptiven Lernumgebungen werden digitale Lernumgebungen für die autonome Lehre gut strukturierter Inhalte verstanden.
Dabei wird zwischen drei Modellen unterschieden: Domänenmodell (Fach), Lernermodell und Didaktikmodell.
Das Domänenmodell behandelt den Inhalt des jeweiligen Faches.
Das Lernermodell betrifft den Zustand des Lernenden und welches Wissen er besitzt.
Das Didaktikmodell schlägt auf Basis der Erkenntnisse des Lernermodells Lernstrategien und Handlungsempfehlungen für Lernende vor. \cite*[S. 17f]{Witt.2020}
\paragraph*{Chatbots}
Lernende können durch Chatbots moderiert und unterstützt werden, wobei sie gleichzeitig Lehrende entlasten. Als Beispiel gibt es einen Chatbot in einem Hochschulforum, welcher Fragestellungen von Lernenden anstelle von Lehrenden beantwortet. \cite*[S. 18ff]{Witt.2020}
\\ \\ \noindent
Darüber hinaus können auch Empfehlungssysteme, Edu-Robots und KI-Schreibbots eingesetzt werden. \cite*[S. 18-21]{Witt.2020}
\subsubsection*{Veränderung der Betreuung}
Anstelle von klassischer Lehre des fachlichen Wissens, sollen Lernende mehr zu kritischem Denken gefördert werden. Das schließt nicht aus, dass Fachkräfte für spezielle Fachrichtungen benötigt werden.
Lernende sollen jedoch mehr dazu gebracht werden Wissen zu generieren, anstatt nur vermittelt zu bekommen.
Daduch wird das Risiko eingegangen, dass nicht zu einer \glqq Allwissenheit\grqq{} ausgebildet wird.
Es wird vermutet, dass sich die Betreuung von Lernenden durch Lehrende im Sinne eines Mentoring verändern wird.
Beim Mentoring ist die Erkennung des Status des Lernenden wichtig, um darauf reaktiv handeln zu können.
Das Mentoring wird in die Phasen Vorbereitung, Lernprozss und Nachbereitung aufgeteilt. \cite*[S. 22f]{Witt.2020}
\begin{itemize}
      \item Vorbereitung: Erhebung der Lerninhalte und Planung der Ausgestaltung des Lernprozesses \cite*[S. 23]{Witt.2020}.
      \item Lernprozess: Ausführung der geplanten individualisierten Aktivitäten zur Aneignung der Lerninhalte \cite*[S. 23]{Witt.2020}.
      \item Nachbereitung: Reflexion des Lernprozesses über Lernerfolg \cite*[S. 23]{Witt.2020}.
\end{itemize}
Intelligente Tutor- und Lernsysteme können zur Verbesserung der Lernergebnisse führen.
Es bestehen allerdings heute noch fehlende Tutoring- und Lernanwendungen mit KI.
Bisher wurden nur Vorhersage von Erfolg und Misserfolg des Lernprozesses eingesetzt bzw. geprüft. \cite*[S. 24]{Witt.2020}

\subsection{Lehren und Lernen über KI in der Hochschulausbildung}
Damit KI effektiv eingesetzt werden kann, muss auch Wissen über KI vermittelt werden.
Dies beginnt bei der Verwendung von KI und geht über die Auswahl geeigneter KI bis hin zur Entwicklung neuer KI.
Die Nachfrage an Fachkräften im Bereich KI wird wahrscheinlich stark wachsen.
Daher sollte der Fokus mehr auf Lehre über KI gesetzt werden. \cite*[S. 26]{Witt.2020}
\\ \noindent
\\ \noindent
Damit die Lehre über KI stattfinden kann, müssen einige Herausforderungen beachtet und gelöst werden.\cite*[S. 27]{Witt.2020}
\begin{itemize}
      \item Es besteht ein uneinheitliches Verständnis über KI. \cite*[S. 27]{Witt.2020}
      \item Aktuell wird \ac{KI} am meisten in Ingenieurwissenschaften verwendet. \cite*[S. 27]{Witt.2020}
      \item Es besteht ein Mangel an fehlenden Standards für digitale Lernangebote. \cite*[S. 27]{Witt.2020}
\end{itemize}

\subsubsection*{Digitale Lehre im Bereich KI}
In Deutschland existieren bereits digitale Plattformen, auf denen auch Lernangebote zum Thema \ac{KI} zu finden sind.
Darunter fallen bspw. oncampus von der TH Lübeck, OPEN VHB von der virtuellen Hochschule Bayern, sowie OER-Content.nrw vom Bundesland Nordrhein-Westfalen.
Insgesamt existieren noch zu wenig Lernangebote zum Thema \ac{KI} auf den Plattformen. In Zukunft sollte es mehr Angebote geben.
Angebote sollten dabei für mehrere Zielgruppen zugänglich und verwendbar sein. Daher sind auch Angebote vonnöten, die keine umgreifendes technisches Wissen voraussetzen.
Zudem sollten Lernangebote offen lizenziert und transparent sein. \cite*[S. 28ff.]{Witt.2020}

\subsection{KI und Ethik in der Hochschulbildung}
KIs als autonome Systeme interagieren mit Menschen und geben Handlungsempfehlungen auf Grundlage von Daten der Menschen.
Aus diesem Grund besteht das Risiko, dass diese Systeme den Menschen Schaden zufügen können.
Dieses Risiko ist ein wichtiger Faktor bei der Berücksichtigung der Ethik von KIs im Zusammenhang mit dem Menschen. \cite*[S. 38]{Witt.2020}

\paragraph*{Qualität der (Trainings-)Daten}
Die Qualität der Daten beeinflusst unmittelbar die Ergebnisse einer \ac{KI}.
Bestehen Tendenzen zu Diskriminierung, Unterrepräsentation, Rassismus oder ähnlicher negativer sozialer Einstellungen in den Trainingsdaten einer \ac{KI}, greift die \ac{KI} diese auf und integriert sie in Entscheidungen und Handlungsempfehlungen.
Daher gilt es, die Qualität der Daten zu sichern. \cite*[S. 39]{Witt.2020}

\paragraph*{Sensibilisierung für KI}
Eine Analyse personenbezogener Daten von Menschen und anschließende Auswertungen können bei Menschen Misstrauen durch entstehende Gefühle von Überwachung und Kontrolle erzeugen.
Weiterhin wird durch die Verwendung unterschiedlichster Daten zur Fremdbestimmung statt Selbstbestimmung tendiert.
Daher müssen Sensibilierungsmaßnahmen getroffen werden, damit die Akzeptanz und das Verständnis gegenüber KIs erhöht werden.
Dies kann bspw. durch frühzeitige Aufklärungsarbeiten über die Funktionsweise und Herausforderungen von KIs erfolgen.
Ein notwendiges Mittel für die Transparenz von KIs ist die Erklärbarkeit, welches im Rahmen der Disziplin Explainable AI erforscht wird. \cite*[S. 39ff.]{Witt.2020}
