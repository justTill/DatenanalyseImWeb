\subsection*{Kernthemen}
\begin{enumerate}
    \item Lehren und Lernen mit KI in der Hochschulausbildung
    \item Lehren und Lernen über KI in der Hochschulausbildung
    \item KI und Ethik im Hochschulkontext
    \item Zukunftsperspektiven für KI in der Hochschulbildung
\end{enumerate}


\subsection{Lehren und Lernen mit KI in der Hochschulausbildung}

Gründe, warum KI eingesetzt werden sollte:
\begin{itemize}
    \item Verwendung von Daten zur Unterstützung von Studierenden bei der Weiterentwicklung des eigenen Lernverhaltens und Lehrenden bei der Verbesserung ihrer Didaktik.
    \item Förderung von kritischem und kreativem Denken
    \item Entdeckung neuer Erkenntnisse durch Analyse von Daten
    \item Vorzeitige Erkennung von Risiko-Studierenden
\end{itemize}

Auflistung von Problemen in Studium und Lehre und wie diese Probleme durch KI-Anwendungen adressiert werden können. Beispiele: Begrenzte Tutorzeit von Lehrenden für Studierende, Lösung virtuelle intelligente Assistenten; Aufwändige Bewertung von Texteinreichungen können durch sog. Automated Essay Scoring automatisiert werden.

\subsubsection*{Learning Analytics}
Learning Analytics als Prozess, um den Prozess des Lernens zu unterstützen. Einteilung von Learning Analytics in drei Varianten: Mikroebene, Mesoeebene und Makroebene. Der Prozess kann unterschiedliche Zielgruppen unterstützen, wie Lernende, Lehrende oder auch gesamte Organisationen. Die Varianten beziehen sich dabei auf den Umfang und zeitlichen Kontext des Lernens. In der Mikroebene können Lernende bspw. im Rahmen einer Lernsitzung oder im Verlaufe eines Kurses unterstützt werden. In der Mesoebene können Lernende über ein gesamtes Semester begleitet und Daten kursübergreifend ausgewertet werden.

Learning Analytics könnte auch für die Erkennung von gefährdeten Lernenden eingesetzt werden. Dabei können auf Grundlage der bisherigen Leistungen Abbrüche oder Noten vorhergesagt werden. Mit

Paper listet Herausforderungen, die bei Learning Analytics beachtet werden müssen. Beispiele: Analysen betrachten meist nur einen Ausschnitt des Lernens; Systeme sollten sozialen Kontext berücksichtigen; Berücksichtigung persönlicher Eigenschaften.

\subsubsection*{Arten von Künstlichen Intelligenzen in der Hochschulbildung}

\paragraph*{Personalisierte adaptive Lernumgebungen (ALU)}
Digitale Lernumgebungen für die autonome Lehre gut strukturierter Inhalte. Trennung in Domänenmodell (Fach), Lernermodell und Didaktikmodell. Das Lernermodell betrifft den Zustand des Lernenden und welches Wissen er besitzt. Das Didaktikmodell schlägt auf Basis des Lernenden Lernstrategien und Handlungsempfehlungen vor.

\paragraph*{Chatbots}
Lernende können durch Chatbots moderiert und unterstützt werden, wobei sie gleichzeitig Lehrende entlasten.

\paragraph*{Empfehlungssysteme}
Empfehlungssysteme sind bislang nur im Auswahlprozess eines Studiengangs im Einsatz.

\paragraph*{Edu-Robots}
Meist Begleitungen oder Ersatz für Lehrende in Kursen.

\paragraph*{KI-Schreibbots}
Generierung einzigartiger Texte im Hochschulkontext. Birgt die Herausforderung von erschwerter Überprüfbarkeit.

\subsubsection*{Veränderung der Betreuung}
Zunehmende Vermittlung von kritischem Denker, statt klassischem Wissen. Risiko durch nicht \glqq Allwissenheit\grqq{} besteht. Beim Mentoring wichtig ist die Erkennung des Status des Lernenden und damit verknüpfte reaktive Handlungen. Aufteilung des Mentoring in die Phasen Vorbereitung, Lernprozss und Nachbereitung.
\begin{itemize}
    \item Vorbereitung: Erhebung der Lerninhalte und Planung der Ausgestaltung des Lernprozesses.
    \item Lernprozess: Ausführung der geplanten individualisierten Aktivitäten zur Aneignung der Lerninhalte.
    \item Nachbereitung: Reflexion des Lernprozesses über Lernerfolg.
\end{itemize}
Intelligente Tutor- und Lernsysteme führen zur Verbesserung der Lernergebnisse. Fehlende Anwendung von KI, bislang nur Vorhersage von Erfolg und Misserfolg des Lernprozesses.

\subsection{Lehren und Lernen über KI in der Hochschulausbildung}
Es wird erwartet, dass die Nachfrage an Fachkräften im Bereich KI stark wachsen wird $\rightarrow$ mehr Fokus auf Lehre über KI setzen.

\paragraph*{Herausforderungen}
\begin{itemize}
    \item Uneinheitliches Verständnis über KI.
    \item Aktuell am meisten Verwendet in Ingenieurwissenschaften, Tendenz interdisziplinär.
    \item Fehlende Standards digitaler Lernangebote.
\end{itemize}

\subsubsection*{Digitale Lehre im Bereich KI}
Anforderung zu mehr digitalen Lernangeboten zu KI. Angebote sollten dabei für mehrere Zielgruppen zugänglich und verwendbar sein. Keine Prämisse für tiefen technischen Hintergrund von Zielgruppen. Lernangebote sollte offen lizenziert sein.

\paragraph*{Deutsche Plattformen für digitale Lernangebote zum Thema KI}
\begin{itemize}
    \item oncampus (TH Lübeck)
    \item Hamburg Open Online University (HOUU)
    \item OPEN VHB (virtuelle Hochschule Bayern)
    \item OpenHPI (Potsdam)
    \item OER-Content.nrw (Nordrhein-Westfalen)
\end{itemize}
$\Rightarrow$ insgesamt sehr kleines Angebot an (kostenlosen) Kursen zum Thema KI.

\subsection{KI und Ethik in der Hochschulbildung}
KI und Ethik Motivation: Autonome Systeme können potenziell Menschen schaden. Daher: Anforderung an Transparenz und Verständnis über autnom getroffene Entscheidungen. Herausforderung in der Erklärbarkeit von KI Algorithmen (Forschung im Bereich \glqq Explainable AI\grqq{}).

\paragraph*{Qualität der Daten} Die Qualität der Daten beeinflusst unmittelbar die Ergebnisse der KI. Gefahr von Diskriminierung / Unterrepräsentation. Neben Algorithmen auch Daten qualitätssichern.

\paragraph*{Sensibilisierung für KI} Analyse der eigenen Daten und anschließende Auswertungen können Misstrauen durch Gefühle von Überwachung und Kontrolle erzeugen. Durch Analyse unterschiedlichster Daten wird zu Fremdbestimmung tendiert. $\rightarrow$ Sensibilierung schaffen um damit Akzeptanz und Verständnis zu erhöhen.