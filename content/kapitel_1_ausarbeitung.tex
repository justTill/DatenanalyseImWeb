% Kapitel
\chapter{Einleitung}

\epigraph{\glqq The user's going to pick dancing pigs over security every time.\grqq\bigskip}
{\textsc{Bruce Schneier}\\ ($\ast$1963)}

\noindent
Das Verfassen einer eigenständigen Ausarbeitung \dots
\indent

\section{Expos\'{e}}
\label{intro}

Bevor eine Ausarbeitung begonnen werden kann, MÜSSEN sich Kan\-di\-dat/-in und Prüfer
auf eine Aufgabenstellung einigen. Als Grundlage dienen hier eigene Ideen des Kandidaten / der
Kandidatin und Vorschläge des Prüfers. Diese Aufgabenstellung MUSS vom Kandidaten/von der Kandidatin
in ein Expos\'{e} überführt werden, welches auf Basis der vorliegenden \LaTeX-Vorlage erstellt
und folgende Form / Struktur haben MUSS:

\begin{description}
 \item[Kontext und Motivation] Eine inhaltliche Einleitung in das Themengebiet der Arbeit
 mit Referenzierung der wichtigsten Literatur sowie eine Motivation, z.\, B. durch Aufarbeitung
 von Literatur getrieben.
 \item[Ziele] Aufzählung und Kurzbeschreibung konkreter Ziele (im Sinne einer Spezifikation)
 \item[Vorgehen] Eine Beschreibung wie die einzelnen Ziele erreicht werden sollen (im Sinne einer Implementierung)
 und wie die Zielerreichung validiert werden soll.
 \item[Projektplan] Ein leichtgewichtiger und realistischer Projektplan basierend auf Meilensteinen
 und untergeordneten Aufgaben, welcher nach Finalisierung des Expos\'{e} durch den Kandidaten/die
 Kandidatin in Gitlab (siehe Abschnitt~\ref{gitlab}) eingepflegt werden MUSS.
\end{description}

Das Expos\'{e} MUSS explizit (Email, direkte mündliche Absprache, etc.) vom Prüfer akzeptiert werden.

Das Expos\'{e} dient im Anschluss als Grundlage für die Einleitung der Ausarbeitung.
Dabei sollen explizit die Abschnitte Kontext und Motivation sowie Ziele in der Einleitung in möglicherweise
überarbeiteter Fassung übernommen werden. Zudem MUSS eine Einleitung eine Erläuterung des weiteren Aufbaus
der Arbeit beinhalten.

\section{Hinweise}

Bitte denken Sie daran, dass Sie die eidesstattliche Erklärung vor Abgabe unterschreiben.

\section{Inhalt}

Die Arbeit MUSS -- neben dem Hauptteil -- nachfolgende Inhalte berücksichtigen:

\begin{itemize}
  \item Titelseite
  \item Eidesstattliche Erklärung
  \item Zusammenfassung und Abstract (Englisch)
  \item Inhaltsverzeichnis, Abbildungsverzeichnis, Tabellenverzeichnis, Abkürzungs\-ver\-zeichnis und Literaturverzeichnis
  \item Einleitung (siehe Abschnitt~\ref{intro})
  \item Aufarbeitung verwandter und relevanter Literatur unter Angabe der konkreten
  Vorgehensweise bei der Literaturrecherche
  \item Kritische Betrachtung der eigenen Arbeit
  \item Fazit bestehend aus einer reflektierten Zusammenfassung und einem Ausblick
\end{itemize}

Weiterhin MÜSSEN, falls anwendbar, vom Prüfer vorgegebene Richtlinien für Coding-Style
und Code-Dokumentation sowie Gestaltung eingehalten werden.

\section{Organisatorisches}

\begin{itemize}
 \item Es gilt die jeweils aktuelle Pr\"ufungsordnung (\S 15 in BMI PO vom 04.08.2010 bzw. \S 15 in MMI PO vom 16.06.2011).
 Lesen Sie aufmerksam die für Sie geltende Prüfungsordnung und richten Sie sich nach den dort
 definierten Vorgaben (es sei denn Sie haben mit dem Prüfer eine Abweichung abgesprochen).
 \item Abzugeben gebunden als Ausdruck (beidseitig bedruckt) und elektronisch als PDF
\end{itemize}

\section{Bewertungskriterien}

Die Bewertung einer Arbeit erfolgt unter anderem auf Grundlage von \textbf{Schwierigkeitsgrad},
\textbf{wissenschaftlicher Arbeitstechnik}, \textbf{ingenieurmäßiger Vorgehensweise}, \textbf{Stil} und \textbf{Form}.
