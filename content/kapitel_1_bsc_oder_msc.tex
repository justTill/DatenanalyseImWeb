% Kapitel
\chapter{Einleitung}

\epigraph{\glqq The user's going to pick dancing pigs over security every time.\grqq\bigskip}
{\textsc{Bruce Schneier}\\ ($\ast$1963)}

\noindent
Das Verfassen einer eigenständigen Bachelor- bzw. Masterarbeit \dots
\indent

\section{Expos\'{e}}
\label{intro}

Bevor eine Bachelor- bzw. Masterarbeit begonnen werden kann, MÜSSEN sich Kan\-di\-dat/-in und Erstprüfer
auf eine Aufgabenstellung einigen. Als Grundlage dienen hier eigene Ideen des Kandidaten/der
Kandidatin und Vorschläge des Erstprüfers. Diese Aufgabenstellung MUSS vom Kandidaten/von der Kandidatin
in ein Expos\'{e} überführt werden, welches auf Basis der vorliegenden \LaTeX-Vorlage erstellt
und folgende Form / Struktur haben MUSS:

\begin{description}
 \item[Kontext und Motivation] Eine inhaltliche Einleitung in das Themengebiet der Arbeit
 mit Referenzierung der wichtigsten Literatur sowie eine Motivation, z.\, B. durch Aufarbeitung
 von Literatur getrieben.
 \item[Wissenschaftlicher Beitrag] Eine Darstellung der Verbesserung der wissenschaftlichen bzw.
 ingenieurwissenschaftlichen Situation in dem Themengebiet.
 \item[Ziele] Aufzählung und Kurzbeschreibung konkreter Ziele (im Sinne einer Spezifikation)
 \item[Vorgehen] Eine Beschreibung wie die einzelnen Ziele erreicht werden sollen (im Sinne einer Implementierung)
 und wie die Zielerreichung validiert werden soll.
 \item[Projektplan] Ein leichtgewichtiger und realistischer Projektplan basierend auf Meilensteinen
 und untergeordneten Aufgaben, welcher nach Finalisierung des Expos\'{e} durch den Kandidaten/die
 Kandidatin in Gitlab (siehe Abschnitt~\ref{gitlab}) eingepflegt werden MUSS.
\end{description}

Das Expos\'{e} MUSS explizit (Email, direkte mündliche Absprache, etc.) vom Prüfer akzeptiert werden.

Das Expos\'{e} dient im Anschluss als Grundlage für die Einleitung der Bachelor- bzw. Masterarbeit.
Dabei sollen explizit die Abschnitte Kontext und Motivation, Wissenschaftlicher Beitrag sowie
Ziele in der Einleitung in möglicherweise überarbeiteter Fassung übernommen werden. Zudem MUSS
eine Einleitung eine Erläuterung des weiteren Aufbaus der Arbeit beinhalten.

\section{Hinweise}

Bitte lassen Sie ein Exemplar des Anmeldeformulars, welches Sie bei der Anmeldung Ihrer Arbeit
im Prüfungsamt haben unterschreiben lassen und auf dem das Abgabedatum vermerkt ist, als zweite
Seite dieses Dokumentes einbinden.

Bei Abgabe zeigen Sie Ihre drei Exemplare im Studienbüro vor und lassen die Abgabe auf den
entsprechenden Formularen eintragen. Zwei Exemplare müssen nun persönlich den beiden Prüfern
übergeben werden. Das dritte Exemplar ist für Sie bestimmt.
Bitte denken Sie auch daran, dass Sie die eidesstattliche Erklärung vor Abgabe unterschreiben.

\section{Inhalt}

Die Arbeit MUSS -- neben dem Hauptteil -- nachfolgende Inhalte berücksichtigen:

\begin{itemize}
  \item Titelseite
  \item Eidesstattliche Erklärung
  \item Zusammenfassung und Abstract (Englisch)
  \item Inhaltsverzeichnis, Abbildungsverzeichnis, Tabellenverzeichnis, Abkürzungs\-ver\-zeichnis und Literaturverzeichnis
  \item Einleitung (siehe Abschnitt~\ref{intro})
  \item Falls im Falle einer Bachelorarbeit vorab eine wissenschaftliche Vertiefung im gleichen Themengebiet
  verfasst wurde, dann MUSS ggf. eine Zusammenfassung zentraler Ergebnisse der wissenschaftlichen Vertiefung
  in die Bachelorarbeit integriert werden. (In Bibtex steht für die Referenzierung von wissenschaftlichen
  Vertiefungen der Typ \verb|@Unpublished| zur Verfügung.)
  \item Aufarbeitung verwandter und relevanter Literatur unter Angabe der konkreten
  Vorgehensweise bei der Literaturrecherche
  \item Validierung und kritische Betrachtung des wissenschaftlichen Beitrags
  \item Fazit bestehend aus einer reflektierten Zusammenfassung und einem Ausblick
\end{itemize}

Weiterhin MÜSSEN, falls anwendbar, vom Erstprüfer vorgegebene Richtlinien für Coding-Style
und Code-Dokumentation sowie Gestaltung eingehalten werden.

\section{Feedback}
Der Erstprüfer gibt einmal ausführlich Feedback zu der Arbeit. Dazu MUSS der/die Kan\-di\-dat/-in
die Arbeit nach ca. $\frac{1}{3}$ Bearbeitungszeit beim Erstprüfer einreichen.

\section{Organisatorisches}

\begin{itemize}
 \item Es gilt die jeweils aktuelle Pr\"ufungsordnung (\S 15 in BMI PO vom 04.08.2010 bzw. \S 15 in MMI PO vom 16.06.2011).
 Lesen Sie aufmerksam die für Sie geltende Prüfungsordnung und richten Sie sich nach den dort
 definierten Vorgaben (es sei denn Sie haben mit dem Erstprüfer eine Abweichung abgesprochen).
 \item Abzugeben gebunden als Ausdruck (beidseitig bedruckt) und elektronisch als PDF
\end{itemize}

\section{Bewertungskriterien}

Die Bewertung einer Arbeit erfolgt unter anderem auf Grundlage von \textbf{Schwierigkeitsgrad},
\textbf{wissenschaftlicher Arbeitstechnik}, \textbf{ingenieurmäßiger Vorgehensweise}, \textbf{Stil} und \textbf{Form}.

\noindent
Das zugehörige Kolloquium wird vor allem basierend auf der \textbf{Wiedergabe der Inhalte}, der \textbf{Foliengestaltung},
\textbf{Stil}, \textbf{Form} und dem anschließenden \textbf{Fachgespräch} bewertet.
